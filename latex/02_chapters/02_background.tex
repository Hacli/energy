\label{sec:background}
Estimating price elasticities of electricity and has been an area of interest for economists for a long time and increasingly so. Deregulation of the electricity market made it important to measure how a broad spectra of consumers (mostly residential, industrial and commercial) react to changing electricity prices.\bigskip \par
Over the past decades the supply side has on one hand become even more volatile following decentralization and the increased share of renewable energy driven by political goals and competitive establishment costs while on the other hand the process of interconnecting electricity markets has help equalize prices between countries and regions. \bigskip \par
In the case of a very inelastic electricity demand \citet{wolak2001impact} outlines how an oligopolistic supply side can capitalize on intraday peaks in demand, however, firms that are able to have a more flexible energy use can face clear advantages in the market while on aggregate making it possible to further increase the share of renewables in the electricity supply.
\par
Thus, it is of importance to conduct better empirical estimations of how consumers respond at a micro level - in particular because time-of-use tariffs and other price instruments are often are often considered as policy tools to incentivize a decreased consumption of power. In the following we highlight the key contributions in this area.

\subsection{Modest price elasticity of demand}
\label{subsec:b_results}
% Previous empirical evidence
% only the ones more closely related to your study!
\citet{patrick2001estimating} were among the first to estimate the demand-side responses to electricity prices for intraday-markets. For firms in England and Wales they find the overall magnitude of the real-time price elasticity of electricity demand to be quite low, though significant. However, for 5 specific industries an elasticity of -0.05 is found on half-hourly consumption. Similarly a very small overall flexibility is found for the Netherlands with a peak elasticity of -0.004 for hour-by-hour total Dutch consumption \citet{lijesen2007real}, however only 15\% of the load is actually traded in the market price. \bigskip \par
Regarding residential electricity demand most estimates of the demand response to price changes of electricity are in the range -2.01 to -0.004 in the short run and -2.25 to -0.04 in the long run as reported in \citet{espey2004turning}, who does a meta analysis of 248 estimates in 36 non-time-of-day studies. With the median being -0.28 in the short run and -0.81 in the long run, elasticities tend to be bigger in the long run which is in accordance with economic reasoning since consumers better can modify their capital stock in the long term.
%Distinguish between estimates for residential consumers and wholesale consumers

\subsection{Heterogeneous effects}
\label{subsec:b_heterogeneity}
Being able to use the average firm-level responses within each code of British Industrial Classification (BIC) \citet{patrick2001estimating} find a substantial heterogeneity across industries not only in terms of the magnitude of the own-price elasticity of electricity demand but also in the within-day patterns of cross-price elasticities. %What are complements and what are substitues
\par
\citet{fan2011price} estimate yearly own-price elasticities for Southern Australia at the aggregated level using a log-linear model for consumption on a half-hourly basis. The authors find heterogeneous effects across quantiles, depending on how extreme the weather is.
\par
Likewise, under extreme prices \citet{alberini2019response} find that Ukrainian households become more attentive and elastic due to price changes and rather complicated tariff schemes.

\subsection{Endogeneity problems}
\label{subsec:b_endogeneity}
Estimating demands-side responses to shifting electricity prices is associated with potential problems of endogeneity as price and consumptions/production are simultaneous. That is, a higher expected demand can push up the overall prices and vice versa. An example of such a mechanism is that increasing demand-side competition can lead to imports of electricity from more expensive energy sources \citep{burke2017price} resulting in price increases. Furthermore, an unobserved factor can influence both prices and demand.
\par
Therefore, for yearly data lagged prices are often included to avoid an omitted variable bias and to combat endogeneity \citep{lijesen2007real}, however, this inclusion creates a dynamic bias instead. This bias is likely to be bigger when estimating long term elasticities \citep{okajima2013estimation}. %\citet{okajima2013estimation} use lagged dependent variables and  as instruments for the resident electricity price (which have also been done by many others) but use an FD GMM model to circumvent the dynamic bias.
\medskip \\
\citet{bonte2015price} were able to use wind speed as an instrument for the spot market price, however, the motivation of this strategy is specific to Germany due to their feed-in-tariff for Renewable Energy Sources that is designed to directly affect the price. Likewise, \citet{graf2013measuring} tried using emissions right and prices for primary energy as instruments but to limited success. For the US \citet{burke2017price} use the state level share of coal and hydro as an instrument for yearly prices as they were generally the two cheaper sources of electricity.

\subsection{Estimation methods}
\label{subsec:b_estimation}
The estimates rely on a wide range of different empirical approaches to an extent that cannot simply be explained by different data structure. Thus, no clear \textit{best practice} has yet been established for this field of research.
\medskip\\
When data is limited to total electricity demand it is common to apply a pooled two-stage-least-squares (P2SLS) regression and either include a time trend \citep{lijesen2007real} or estimate the elasticity year-by-year \citep{bonte2015price}.
\par
For more dis-aggregated data different methods can be utilized. Unobserved heterogeneity must be accounted for, usually by including unit-specific time-constant unobserved effects. On household level data one such option is to use a Seemingly Unrelated Regression Equations (SUR/SURE) model \citep{vesterberg2014residential}, however, \citet{alberini2019response} still use P2SLS with a rich set of background variables.
\par
Lastly, identification strategies that rely on using past prices as instruments often rely on Dynamic Panel Estimation using Generalized Methods of Moments (GMM) \citep{genc2016measuring} and intertemporal substitution can be modelled using Quasi Maximum Likelihood \citep{wolak2001impact}.
