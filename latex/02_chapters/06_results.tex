\label{sec:results}
% Consequences and implications for the public policy from different expected results.
\subsection{Results for wholesale}
\label{subsec:r_wholesale}
For wholesale consumers the baseline specification (\ref{eq:baseline}) is first estimated separately for each hour of the business day to identify peak, off-peak, and the shoulder hours. Based on these the peak-period is defined as the five consecutive hours 11am-3pm for which the estimated elasticity $\hat{\varepsilon}$ is below $-.045$, wile the off-peak period is defined as the five consecutive hours 12am-4am where $\hat{\varepsilon}$ is greater than $-.030$. The hours on each side of these intervals are classified as shoulder periods. For non-business days none of these classifications are used because the estimated elasticities do not vary much, they are all of small small magnitude ($\hat{\varepsilon}\geq-.03$)  and even insignificant for several hours of the day.

\begin{table}[H]
\begin{threeparttable}
  \centering
  \caption{log wholesale electricity consumption (REIV)}
  \footnotesize
    \begin{tabular}{lcccc}
        \begin{tabular}{lcccc}\toprule
                    &(1) Peak: 11-15   &(2) Off-peak: 00-04   &(3) Shoulder   &(4) Non-business day   \\
                    &        b/se   &        b/se   &        b/se   &        b/se   \\
\midrule
log spot price      &     -0.0484***&     -0.0266***&     -0.0333** &     -0.0189*  \\
                    &    (0.0163)   &    (0.0094)   &    (0.0149)   &    (0.0099)   \\
log wholesale meters&      0.1578***&      0.1422***&      0.1255***&      0.1424***\\
                    &    (0.0375)   &    (0.0399)   &    (0.0332)   &    (0.0375)   \\
Temperature         &     -0.0036***&     -0.0014** &     -0.0022***&     -0.0038***\\
                    &    (0.0008)   &    (0.0006)   &    (0.0004)   &    (0.0006)   \\
Temperature squared &      0.0002***&      0.0001***&      0.0001***&      0.0002***\\
                    &    (0.0000)   &    (0.0000)   &    (0.0000)   &    (0.0000)   \\
Daytime             &               &               &      0.0198***&      0.0966***\\
                    &               &               &    (0.0052)   &    (0.0085)   \\
Time variables      &         Yes   &         Yes   &         Yes   &         Yes   \\
\midrule
\(R^2\) within      &      0.3614   &      0.1576   &      0.5797   &      0.1414   \\
\(R^2\) between     &      0.9492   &      0.9140   &      0.9375   &      0.9250   \\
Number of groups    &          48   &          48   &          48   &          48   \\
Obs. per group      &       3,675   &       3,675   &      13,178   &       8,660   \\
\bottomrule\end{tabular}

    \end{tabular}
    \begin{tablenotes}
    \item Standard errors are clustered at grid level and reported in parentheses. * p<0.10, ** p<0.05, *** p<0.01.
    \item Log spot price is instrumented for by wind power prognosis for the same region.
     \end{tablenotes}
  \label{tab:ws_preferred}
\end{threeparttable}
\end{table}
\noindent
For the peak-hours 11-15 on business days wholesale of electricity is estimated to decrease with 5 percent when the spot price doubles all other things equal, as can be noted from column 1 in table \ref{tab:ws_preferred}. The difference between business and non-business days is quite outspoken; wholesale consumers are 1,5 times as responsive at peak on business-days compared to the average on non-business days. Weather and daytime controls also have significant effect on energy consumption. An increase in temperature of  $1^{\circ}$C translates into a small decrease of 0.3 percent lower electricity consumption - except at extreme temperatures (cold or warm) where consumption increases slightly.   

\subsection{Results for households and small companies}
The estimation of \eqref{eq:baseline} is reported below in table \ref{tab:r_region}. Pooling across all days we find that a 100 percent increase in the spot price causes a decrease consumption of about 2.75 percent - which appears to be driven by reductions on business days. This suggests that people are either may be less. oive on .. The overall electricity elasticity is
\label{subsec:r_households}
\begin{table}[H]
\begin{threeparttable}
  \centering
  \caption{log retail electricity consumption by region, hours 17-19 (REIV)}
  \label{tab:r_region}
  \footnotesize
    \begin{tabular}{lccccc}\toprule
                    &     (1) All   &(2) Business day   &(3) Non-business day   &     (4) DK1   &     (5) DK2   \\
                    &        b/se   &        b/se   &        b/se   &        b/se   &        b/se   \\
\midrule
log spot price      &     -0.0275***&     -0.0354***&     -0.0354***&     -0.0292***&     -0.0305***\\
                    &    (0.0056)   &    (0.0059)   &    (0.0059)   &    (0.0061)   &    (0.0092)   \\
Share time-of-use tariff&     -0.0406***&     -0.0350***&     -0.0350***&               &     -0.0069   \\
                    &    (0.0101)   &    (0.0110)   &    (0.0110)   &               &    (0.0148)   \\
Oct-Mar (Radius only)&      0.0517***&      0.0531***&      0.0531***&               &      0.0204   \\
                    &    (0.0119)   &    (0.0117)   &    (0.0117)   &               &    (0.0190)   \\
log retail meters   &      0.9077***&      0.9253***&      0.9253***&      0.8960***&      1.0056***\\
                    &    (0.0376)   &    (0.0351)   &    (0.0351)   &    (0.0369)   &    (0.0340)   \\
Temperature         &     -0.0035***&     -0.0046***&     -0.0046***&     -0.0039***&     -0.0037***\\
                    &    (0.0004)   &    (0.0005)   &    (0.0005)   &    (0.0005)   &    (0.0007)   \\
Temperature squared &      0.0001***&      0.0001***&      0.0001***&      0.0001***&      0.0001** \\
                    &    (0.0000)   &    (0.0000)   &    (0.0000)   &    (0.0000)   &    (0.0000)   \\
Daytime             &     -0.0409***&     -0.0449***&     -0.0449***&     -0.0338***&     -0.0582***\\
                    &    (0.0030)   &    (0.0030)   &    (0.0030)   &    (0.0018)   &    (0.0040)   \\
Time variables      &         Yes   &         Yes   &         Yes   &         Yes   &         Yes   \\
\midrule
\(R^2\) within      &      0.8086   &      0.8152   &      0.8152   &      0.7923   &      0.8866   \\
\(R^2\) between     &      0.9930   &      0.9933   &      0.9933   &      0.9927   &      0.9963   \\
Number of groups    &          48   &          48   &          48   &          39   &           9   \\
Obs. per group      &       3,288   &       2,205   &       2,205   &       3,288   &       3,288   \\
\bottomrule\end{tabular}

    \begin{tablenotes}
    \item Standard errors are clustered at grid level and reported in parentheses. * p<0.10, ** p<0.05, *** p<0.01.
    \item Log spot price is instrumented for by wind power prognosis for the same region.
  \end{tablenotes}
\end{threeparttable}
\end{table}
%It may be surprising that 

We examine the grid company Radius seperately. Radius operates in the Copenhagen metropolitan and its flex-settled customers (households and small companies) are charged a Time-of-Use (TOU) tariff of 0.835 DKK (0.112 EUR) for the hours 17-19 from October until March and 0.3236 DKK (0.043 EUR) otherwise. The estimated effect of this tariff is found to be a decrease in electricity demand of 1.9 percent when the spot price doubles. However, on business days the smaller effect of 1.4 percent is only statistically significant at the 10\% level while the decrease is 4.4 percent. on non-business days. Table \ref{tab:r_radius} shows pooled 2SLS estimates of electricity consumption for households and small companies in Radius for the hours 17, 18, and 19. The estimation results also show a small elasticity for the hourly spot price which is instrumented for by wind power prognosis for DK2 and DK1. This is despite that for two of the three years none of the consumers pay the spot market price rather than a fixed price. %Jeg har ændret da man jo betaler for andet end bare gennemsnitsprisen ligesom det er forskelligt om du betaler gns. af kvartal eller året. 
\begin{table}[H]
\begin{threeparttable}
  \vspace{-0.0cm}
  \centering
  \caption{log retail electricity consumption in Radius, hours 17-19 (P2SLS)}
  \label{tab:r_radius}
      \footnotesize
  \begin{tabular}{lccc}
        \begin{tabular}{lccc}\toprule
                    &(1) All days   &(2) Business days   &(3) Non-business days   \\
                    &        b/se   &        b/se   &        b/se   \\
\midrule
log spot price      &     -0.0184** &     -0.0251***&      0.0061   \\
                    &    (0.0076)   &    (0.0081)   &    (0.0179)   \\
Share time-of-use tariff&     -0.0219***&     -0.0137*  &     -0.0408** \\
                    &    (0.0081)   &    (0.0080)   &    (0.0174)   \\
log retail meters   &     -1.6012*  &     -1.2890   &      0.3618   \\
                    &    (0.8606)   &    (0.9208)   &    (1.6832)   \\
Temperature         &     -0.0029***&     -0.0040***&     -0.0026** \\
                    &    (0.0006)   &    (0.0007)   &    (0.0013)   \\
Temperature squared &      0.0000   &      0.0000   &     -0.0000   \\
                    &    (0.0000)   &    (0.0000)   &    (0.0000)   \\
Daytime             &     -0.0450***&     -0.0450***&     -0.0250   \\
                    &    (0.0104)   &    (0.0108)   &    (0.0198)   \\
Time variables      &         Yes   &         Yes   &         Yes   \\
\midrule
Adj. \(R^2\)        &      0.9462   &      0.9587   &      0.9297   \\
Observations        &       3,288   &       2,205   &       1,083   \\
\bottomrule\end{tabular}

  \end{tabular}
    \begin{tablenotes}
        \item  Standard errors are clustered at grid level and reported in parentheses. * p<0.10, ** p<0.05, *** p<0.01.
         \item Log spot price is instrumented using the wind power prognosis.
    \end{tablenotes}
  \vspace{-0.0cm}
  \end{threeparttable}
\end{table}


\subsection{The validity of instruments}
\label{subsec:r_validity}

\begin{table}[H]
\begin{threeparttable}
  \centering
  \caption{Reduced form of log spot price for DK1, business days, hours 11-15 (POLS)}
  \label{tab:reduced_form_price_dk1}
  \footnotesize
  \begin{tabular}{lcccc}
         \begin{tabular}{lcccc}\toprule
                    &(1) 3 instruments   &(2) DK1 and DK2   &     (3) DK1   &    (4) None   \\
                    &        b/se   &        b/se   &        b/se   &        b/se   \\
\midrule
Wind power prognosis same region&     -0.0920***&     -0.0951***&     -0.1617***&               \\
                    &    (0.0137)   &    (0.0130)   &    (0.0079)   &               \\
Wind power prognosis other region&     -0.2727***&     -0.2724***&               &               \\
                    &    (0.0478)   &    (0.0479)   &               &               \\
Wind power prognosis for Sweden&     -0.0048   &               &               &               \\
                    &    (0.0057)   &               &               &               \\
log wholesale meters&     -0.6420   &     -0.6021   &     -1.0020   &               \\
                    &    (0.9546)   &    (0.9505)   &    (0.9144)   &               \\
Temperature         &     -0.0236***&     -0.0238***&     -0.0235***&      0.0001   \\
                    &    (0.0033)   &    (0.0033)   &    (0.0034)   &    (0.0001)   \\
Temperature squared &      0.0008***&      0.0008***&      0.0008***&     -0.0000***\\
                    &    (0.0001)   &    (0.0001)   &    (0.0001)   &    (0.0000)   \\
Time variables      &         Yes   &         Yes   &         Yes   &         Yes   \\
\midrule
Adj. \(R^2\)        &      0.4604   &      0.4605   &      0.4550   &      0.9480   \\
Observations        &       3,675   &       3,675   &       3,675   &       3,675   \\
\bottomrule\end{tabular}

  \end{tabular}
    \begin{tablenotes}
        \item Robust standard errors are in parentheses. * p<0.10, ** p<0.05, *** p<0.01.
    \end{tablenotes}
\end{threeparttable}
\end{table}

Test for endogeneity and overidentifying restrictions as shown in in appendix \ref{app:statistical_tests}). \citep{statacorp2017stata}.
\medskip\\


\subsection{Heterogeneity and robustness}
\label{subsec:r_robustness}


Heterogeneous effects
\begin{figure}[H]
  \centering
  \caption{Wholesale elasticity by hour}
  \label{fig:ws_elasticity_hour}
\end{figure}

\begin{figure}[H]
  \centering
  \caption{Wholesale peak-elasticity by log grid size}
  \label{fig:ws_elasticity_grid}
\end{figure}


\subsection{Discussion}
\label{subsec:r_discussion}
We obtain estimates of the short term elasticity of demand that are in the range of -0.019 to -0.048 for wholesale consumers and -0.018 to -0.025 for retail consumers. These are bigger than those obtained by \cite{lijesen2007real} but around the same size as \cite{wolak2001impact} finds. Part of difference has to with few having access to as high frequency data.  %High frequency => lower estimated elasticity.  In ger within the ranges of previous studies as reported in the meta-analysis by \cite{labandeira2017meta} that finds an overall elasticity of
All though estimates are statistically significant their economic significance is not as big. They all point to an inelastic electricity demand even for the most elastic part of the market. This suggests that the prospects of using decentralized interventions such as demand response programs are limited. From the small estimated effect on the time-of-use tariff in Radius similar conclusions can be conjectured. It should still be although they may still matter at peak demand where prices rise quite dramatically and small decreases in consumption still matter.
\smallskip \\

The time-of-use tariff did seem to have an effect outside of the area that was actually affected by the tariff. This could be a response from knowing that electricity demand and thus emissions from production are high in the peak period. This extra piece of information could be the driver of the result rather than an effect from changed prices. So even though people do not appear to respond much to prices of electricity this could just be due to little available information. It could, however, also be the case that people have not had enough time to adjust their behaviour, which can be costly in terms of utility. Adjustment time may also depend on the implementation of the program i.e. the extent to which consumers were aware of the tariff introduction.
\smallskip \\

There could still be decentralized solutions to the problem. One obvious solution would be to limit the number of contracts where consumers pay a fixed price and increase contracts with flexible settlement. The EU are in the process of implementing a great deal of policies moving in this direction. There is, however, little empirical support of what difference implementation could do without the provision of additional information. Given that people have limited cognitive capacity it could be useful to provide cost examples of using different electrical appliances during peak compared to off-peak or shoulder periods.
Another concern here is that this could also have the opposite effect if the price provided then is perceived as too small to matter rather than exacerbate prices. If people rely on heuristics this could likely be a "harmful" rule of thumb.
\smallskip \\

%Section on some of the experimental evidence out there.
There is much experimental research devoted to look into getting people to conserve energy using non-standard economic tools because the standard tools does not appear to alter behaviour. This paper highlights the importance of these results. Examples include \cite{allcott2011social} where US consumers are informed about how their own consumption of energy compares to that of their neighbours which especially causes those with a relatively high consumption to adjust it to a level closer to that of their neighbours thereby conforming to social norms. This is also an argument in favour of decentralized solutions that focus on moves of the demand curve rather than along it. \cite{kirschen2003demand} argues that electricity is perceived as necessary, but it may be needed to change the perception of what constitutes "normal consumption". From descriptive analysis we note that consumption is much lower in summer so potential to reduce in wintertime too despite higher requirements for electrical heating and lighting.
\cite{allcott2014short} find . decaying effect.


%Måske det næste skal sættes ind et andet sted i stedet.
Another example is \citep{saele2011demand} where they use information in combination with a DR-mechanism. Authors find that costumers respond more than in other studies and conclude that it especially has potential for consumers with electrical heating which is not of much relevance to Denmark.\smallskip \\

Demand responses may not be the most easily implemented and effects may not be big enough to really make a difference. These initiatives may be cost-effective but there is also limited evidence of how persistent the effects are over time. \cite{allcott2014short} among others find that effects are decaying after treatment has ended. \smallskip \\


Given how costly it can be for consumers to alter consumption behavior there may be a bigger need for more centralized solutions to the issue.One option could be to directly affect and alter the supplying capacities.
Increased market integration may satisfy many of these requirements. In section \ref{subsec:t_EU} it is described how better integration in terms of greater connecting capacities across the current price regions could lead to less price volatility in spite of more reliance on intermittent renewables because of a more optimal energy mix. This corresponds to diversification of the generation portfolio. Expanding grid boundaries can make an electricity production that relies on intermittent renewable sources more stable. This almost corresponds to invest across a market index which diminishes individual risk from each producer. It would for allow for complementary production capacities among the energy producers. Hydro and wind energy for example complement each other well;  hydro can be deployed when the wind does not blow and then cheap wind power can be used to pump water reserves full again. Initiatives in this direction are already being taken at the European level with the 'Clean Energy for all Europeans' package consisting of 8 legislative acts and the renewable energy directive.

%Lav en overgang
Technology has thus far prevented "smart" solutions but an increasing number of countries are rolling out smart meters that allow for integration into a smart grid. A smart meter can allows for remote metering and can show current consumption. In similar fashion a smart device is one where its electricity consumption can be changed automatically in response to the electricity price.
\cite{biggar2014economics} reports how end-of-use consumers are committing to make heir devices capable of being responsive to the wholesale market. This implies more integration of retail consumers into the wholesale market which according to out results suggests could be a way get higher elasticity. The current system is build to over cater to peak demand without load shedding. \todo{God sætning, men skal den være her? }

** LILLE KONKLUSION HER **


*. %Add section with how other kinds of demand has shifted in response to the climate crisis - increased focus on name some food..


\subsection{Possible extensions}
\label{subsec:r_extensions}
One possible extension could be to include grid-specific effects other than the random constant term. This is feasible yet cumbersome. The motivation being that one can expect a exists great variation between companies in terms of size, distribution of customers (residential and commercial), industry-intensity. Both at a certain point in time and regarding the time patterns. In our estimation results the time trend does not carry much explanatory power given the other controls, however, this can be due to effects of different direction for different grids covering different areas of the country.
\medskip\\
A very tractable extension would be to use micro data which would allow to control for compositional changes in the presence of heterogeneous consumers. Similarly more detailed data would allow for an exploration of further heterogeneities in terms of who is more or less responsive to electricity prices. This would be useful for both the wholesale market where it could be interesting to look at businesses in different industries while for the retail consumers it would be interesting to explore how educational level affects price responsiveness.
