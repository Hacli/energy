\label{sec:results}
% Consequences and implications for the public policy from different expected results.
\subsection{Results for wholesale}
\label{subsec:r_wholesale}
For the peak-hours 11-15 on business days wholesale of electricity is estimated to decrease with 5.5 pct. when the spot price doubles, other things equal (Table \ref{tab:ws_preferred}).
\begin{table}[H]
  \vspace{-0.0cm}
  \centering
  \caption{log wholesale electricity consumption (REIV)}
  \footnotesize
    \begin{tabular}{lcccc}
      \toprule
        \begin{tabular}{lcccc}\toprule
                    &(1) Peak: 11-15   &(2) Off-peak: 00-04   &(3) Shoulder   &(4) Non-business day   \\
                    &        b/se   &        b/se   &        b/se   &        b/se   \\
\midrule
log spot price      &     -0.0484***&     -0.0266***&     -0.0333** &     -0.0189*  \\
                    &    (0.0163)   &    (0.0094)   &    (0.0149)   &    (0.0099)   \\
log wholesale meters&      0.1578***&      0.1422***&      0.1255***&      0.1424***\\
                    &    (0.0375)   &    (0.0399)   &    (0.0332)   &    (0.0375)   \\
Temperature         &     -0.0036***&     -0.0014** &     -0.0022***&     -0.0038***\\
                    &    (0.0008)   &    (0.0006)   &    (0.0004)   &    (0.0006)   \\
Temperature squared &      0.0002***&      0.0001***&      0.0001***&      0.0002***\\
                    &    (0.0000)   &    (0.0000)   &    (0.0000)   &    (0.0000)   \\
Daytime             &               &               &      0.0198***&      0.0966***\\
                    &               &               &    (0.0052)   &    (0.0085)   \\
Time variables      &         Yes   &         Yes   &         Yes   &         Yes   \\
\midrule
\(R^2\) within      &      0.3614   &      0.1576   &      0.5797   &      0.1414   \\
\(R^2\) between     &      0.9492   &      0.9140   &      0.9375   &      0.9250   \\
Number of groups    &          48   &          48   &          48   &          48   \\
Obs. per group      &       3,675   &       3,675   &      13,178   &       8,660   \\
\bottomrule\end{tabular}

    \end{tabular}
    \text{Cluster robust standard errors are in parentheses. * p<0.10, ** p<0.05, *** p<0.01.}
    \text{Log spot price is instrumented for by wind power prognosis for the same and the other region.}
  \label{tab:ws_preferred}
  \vspace{-0.0cm}
\end{table}


\subsection{Results for households and small companies}
\label{subsec:r_households}
For the grid company Radius operating in the Copenhagen metropolitan area flex-settled customers (households and small companies) are charged a Time-of-Use (TOU) tariff of 0.835 DKK (0.112 EUR) for the hours 17-19 from October-March as opposed to 0.3236 DKK otherwise (0.043 EUR). The estimated effect of this tariff is found to be a decrease in electricity demand of 1.9 pct. However, on business days the smaller effect of 1.4 pct. is only statistically significant at the 10\% level while the decrease is 4.4 pct. on non-business days. Table \ref{tab:r_radius} shows pooled 2SLS estimates of electricity consumption for households and small companies in Radius for the hours 17, 18, and 19. The estimation results also show a small elasticity for the hourly spot price which is instrumented for by wind power prognosis for DK2 and DK1. This is despite that for two of the three years none of the consumers pay the spot market price rather than a quarterly average.
\begin{table}[H]
  \vspace{-0.0cm}
  \centering
  \caption{log retail electricity consumption in Radius, hours 17-19 (P2SLS)}
  \label{tab:r_radius}
  \footnotesize
        \begin{tabular}{lccc}\toprule
                    &(1) All days   &(2) Business days   &(3) Non-business days   \\
                    &        b/se   &        b/se   &        b/se   \\
\midrule
log spot price      &     -0.0184** &     -0.0251***&      0.0061   \\
                    &    (0.0076)   &    (0.0081)   &    (0.0179)   \\
Share time-of-use tariff&     -0.0219***&     -0.0137*  &     -0.0408** \\
                    &    (0.0081)   &    (0.0080)   &    (0.0174)   \\
log retail meters   &     -1.6012*  &     -1.2890   &      0.3618   \\
                    &    (0.8606)   &    (0.9208)   &    (1.6832)   \\
Temperature         &     -0.0029***&     -0.0040***&     -0.0026** \\
                    &    (0.0006)   &    (0.0007)   &    (0.0013)   \\
Temperature squared &      0.0000   &      0.0000   &     -0.0000   \\
                    &    (0.0000)   &    (0.0000)   &    (0.0000)   \\
Daytime             &     -0.0450***&     -0.0450***&     -0.0250   \\
                    &    (0.0104)   &    (0.0108)   &    (0.0198)   \\
Time variables      &         Yes   &         Yes   &         Yes   \\
\midrule
Adj. \(R^2\)        &      0.9462   &      0.9587   &      0.9297   \\
Observations        &       3,288   &       2,205   &       1,083   \\
\bottomrule\end{tabular}

    \text{Robust standard errors are in parentheses. * p<0.10, ** p<0.05, *** p<0.01.}
    \text{Log spot price is instrumented for by wind power prognosis.}
  \vspace{-0.0cm}
\end{table}


\subsection{The validity of instruments}
\label{subsec:r_validity}

\begin{table}[H]
  \centering
  \caption{Reduced form of log spot price for DK1 (POLS)}
  \label{tab:reduced_form_price_dk1}
  \footnotesize
        \begin{tabular}{lcccc}\toprule
                    &(1) 3 instruments   &(2) DK1 and DK2   &     (3) DK1   &    (4) None   \\
                    &        b/se   &        b/se   &        b/se   &        b/se   \\
\midrule
Wind power prognosis same region&     -0.0920***&     -0.0951***&     -0.1617***&               \\
                    &    (0.0137)   &    (0.0130)   &    (0.0079)   &               \\
Wind power prognosis other region&     -0.2727***&     -0.2724***&               &               \\
                    &    (0.0478)   &    (0.0479)   &               &               \\
Wind power prognosis for Sweden&     -0.0048   &               &               &               \\
                    &    (0.0057)   &               &               &               \\
log wholesale meters&     -0.6420   &     -0.6021   &     -1.0020   &               \\
                    &    (0.9546)   &    (0.9505)   &    (0.9144)   &               \\
Temperature         &     -0.0236***&     -0.0238***&     -0.0235***&      0.0001   \\
                    &    (0.0033)   &    (0.0033)   &    (0.0034)   &    (0.0001)   \\
Temperature squared &      0.0008***&      0.0008***&      0.0008***&     -0.0000***\\
                    &    (0.0001)   &    (0.0001)   &    (0.0001)   &    (0.0000)   \\
Time variables      &         Yes   &         Yes   &         Yes   &         Yes   \\
\midrule
Adj. \(R^2\)        &      0.4604   &      0.4605   &      0.4550   &      0.9480   \\
Observations        &       3,675   &       3,675   &       3,675   &       3,675   \\
\bottomrule\end{tabular}

    \text{Robust standard errors are in parentheses. * p<0.10, ** p<0.05, *** p<0.01.}
    \text{Business days, hours 11-15. Baseline: year 2016 and each hour for December.}
\end{table}
Test for endogeneity (Appendix \ref{app:endogeneity}) and overidentifying restrictions (Appendix \ref{app:overidentifying}). \citep{statacorp2017stata}.
\medskip\\


\subsection{Heterogeneity and robustness}
\label{subsec:r_robustness}


Heterogeneous effects
\begin{figure}[H]
  \centering
  \caption{Wholesale elasticity by hour}
  \label{fig:ws_elasticity_hour}
\end{figure}

\begin{figure}[H]
  \centering
  \caption{Wholesale peak-elasticity by log grid size}
  \label{fig:ws_elasticity_grid}
\end{figure}


\subsection{Discussion}
\label{subsec:r_discussion}

The estimated elasticities that we obtain are in size and statistical significance close to what others have obtained\textbf{ (insert references)}.  they all point to the same thing - namely that even for the most elastic part of the market electricity demand is inelastic. This suggests that the prospects of using demand response programs are limited.

Decentralized solutions should focus on moves of the demand curve rather than along.
The small magnitude of the effect and how weak evidence from the Time-of-use introduced in Radius' grid points in the same direction.
\par

The time-of-use tariff did seem to have an effect outside of the area that was actually affected by the tariff. This could be a response from knowing that electricity demand and thus emissions from production are high in the peak period. This extra piece of information could be the driver of the result rather than an effect from changed prices. So even though people do not appear to respond much to prices of electricity this could just be due to little available information.
\par

There could still be decentralized solutions to the problem. One obvious solution would be to limit the number of contracts where consumers pay a fixed price and increase contracts with flexible settlement and then do this in combination with programs that provide more information to what the price of electricity is, but also the cost of using a given appliance during peak compared to off-peak or shoulder periods. Legislation such that price examples are provided - even though this could burden the producers and ultimately result in higher prices for the appliance itself. Another concern here is that this could also have the opposite effect if the price provided then is perceived as too small to matter rather than exacerbate prices. This would likely be the case if people rely on heuristics.
\par

%Section on some of the experimental evidence out there.
Furthermore there is much experimental research devoted to look into getting people to conserve energy because consumers standard economic tools does not appear to have much of an effect. This paper highlights the importance of these results. Examples include \citep{allcott2011social} where US consumers are informed about how their own consumption of energy compares to that of their neighbours which especially causes those with a relatively high consumption to adjust it to a level closer to that of their neighbours thereby conforming to social norms. This points to a similar solution of trying to change what is perceived as "normal consumption" and how electricity is perceived as a necessary good. From descriptive analysis we note that consumption is much lower in summer so potential to reduce in wintertime too despite higher requirements for electrical heating and lighting. Another example is \citep{saele2011demand} where they use information in combination with a DR-mechanism. Authors find that costumers respond more than in other studies and conclude that it especially has potential for consumers with electrical heating which is not of much relevance to Denmark.
\par

Demand side management may not be the most easiliy implemented and effects may not be enough to really make a difference. These intiatives may be cost-effective but there is also limited evidence of how persistent the effects are over time.

\par %Decentralized bit:
Given how costly it can be for the consumer to alter their consumption behaviours there may be a bigger need for more centralized solutions to the issue.

One option could be to directly affect and alter the supplying capacities.

Increased market integration may actually satisfy many of these requirements. As discussed  More integrated electrical grids in terms of greater connecting capacities across the current price regions. This could lead to less price volatility in spite of more reliance on intermittent renewables. This corresponds to diversification of generation portfolio.  Eventhough some of the generated electricity is lost from transportation over large distances improvements to the connecting capacities could still smooth out price differences across grids. This could make an electricity production that relies on intermittent renewable sources more stable. This would almost correspond to invest across a market index which would make individual price risk from each producer smaller. It would for allow for complementary production capacities among the energy producers. This is already happening but could be part of getting rid of fossil reliant energy production. Hydro and wind energy for example complement each other well. Hydro makes a great reserve which can be deployed in all kinds of weather so useful when the wind does not blow and then cheap wind power can be used to pump water reserves full again. Initiatives in this direction are already being taken at the European level with the 'Clean Energy for all Europeans' package consisting of 8 legislative acts and the renewable energy directive.

This alone may not be enough. This may not be as feasible because it relies on



\subsection{Possible extensions}
\label{subsec:r_extensions}
One possible extension could be to include grid-specific effects other than the random constant term. This is feasible yet cumbersome. The motivation being that one can expect a exists great variation between companies in terms of size, distribution of customers (residential and commercial), industry-intensity. Both at a certain point in time and regarding the time patterns. In our estimation results the time trend does not carry much explanatory power given the other controls, however, this can be due to effects of different direction for different grids covering different areas of the country.
\medskip\\
A very tractable extension would be to use micro data which would allow to control for compositional changes in the presence of heterogeneous consumers. Similarly more detailed data would allow for an exploration of further heterogeneities in terms of who is more or less responsive to electricity prices. This would be useful for both the wholesale market where it could be interesting to look at businesses in different industries while for the retail consumers it would be interesting to explore how educational level affects price responsiveness.
