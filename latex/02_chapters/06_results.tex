\label{sec:results}
% Consequences and implications for the public policy from different expected results.
\subsection{Results for wholesale}
\label{subsec:r_wholesale}
For the peak-hours 12-15 on business days wholesale of electricity is estimated to decrease with 5.5 pct. when the spot price doubles, other things equal (Table \ref{tab:ws_preferred}).
\begin{table}[H]
  \vspace{-0.0cm}
  \centering
  \caption{log wholesale electricity consumption (REIV)}
  \footnotesize
    \begin{tabular}{lcccc}
      \toprule
        \begin{tabular}{lcccc}\toprule
                    &(1) Peak: 11-15   &(2) Off-peak: 00-04   &(3) Shoulder   &(4) Non-business day   \\
                    &        b/se   &        b/se   &        b/se   &        b/se   \\
\midrule
log spot price      &     -0.0484***&     -0.0266***&     -0.0333** &     -0.0189*  \\
                    &    (0.0163)   &    (0.0094)   &    (0.0149)   &    (0.0099)   \\
log wholesale meters&      0.1578***&      0.1422***&      0.1255***&      0.1424***\\
                    &    (0.0375)   &    (0.0399)   &    (0.0332)   &    (0.0375)   \\
Temperature         &     -0.0036***&     -0.0014** &     -0.0022***&     -0.0038***\\
                    &    (0.0008)   &    (0.0006)   &    (0.0004)   &    (0.0006)   \\
Temperature squared &      0.0002***&      0.0001***&      0.0001***&      0.0002***\\
                    &    (0.0000)   &    (0.0000)   &    (0.0000)   &    (0.0000)   \\
Daytime             &               &               &      0.0198***&      0.0966***\\
                    &               &               &    (0.0052)   &    (0.0085)   \\
Time variables      &         Yes   &         Yes   &         Yes   &         Yes   \\
\midrule
\(R^2\) within      &      0.3614   &      0.1576   &      0.5797   &      0.1414   \\
\(R^2\) between     &      0.9492   &      0.9140   &      0.9375   &      0.9250   \\
Number of groups    &          48   &          48   &          48   &          48   \\
Obs. per group      &       3,675   &       3,675   &      13,178   &       8,660   \\
\bottomrule\end{tabular}

    \end{tabular}
    \text{Cluster robust standard errors are in parentheses. * p<0.10, ** p<0.05, *** p<0.01.}
    \text{Log spot price is instrumented for by wind power prognosis for the same and the other region.}
  \label{tab:ws_preferred}
  \vspace{-0.0cm}
\end{table}


\subsection{Results for households and small companies}
For the grid company Radius operating in the Copenhagen metropolitan area flex-settled customers (households and small companies) are charged a Time-of-Use (TOU) tariff of 0.835 DKK (0.112 EUR) for the hours 17-19 from October-March as opposed to 0.3236 DKK otherwise (0.043 EUR). The estimated effect of this tariff is found to be a decrease in electricity demand of 1.9 pct. However, on business days the smaller effect of 1.4 pct. is only statistically significant at the 10\% level while the decrease is 4.4 pct. on non-business days. Table \ref{tab:hh_17-19} shows pooled 2SLS estimates of electricity consumption for households and small companies in Radius for the hours 17, 18, and 19. The estimation results also show a small elasticity for the hourly spot price which is instrumented for by wind power prognosis for DK2 and DK1. This is despite that for two of the three years none of the consumers pay the spot market price rather than a quarterly average.
\label{subsec:r_households}
\begin{table}[H]
  \vspace{-0.0cm}
  \centering
  \caption{log retail electricity consumption in Radius, hours 17-19 (2SLS)}
  \footnotesize
    \begin{tabular}{lccc}
      \toprule
                            &(1) All days   &(2) Business days   &(3) Non-business days   \\
                    &        b/se   &        b/se   &        b/se   \\
\midrule
log spot price      &    -0.01597** &    -0.02624***&    -0.00515   \\
                    &   (0.00734)   &   (0.00803)   &   (0.01823)   \\
Time-of-use tariff  &    -0.01907** &    -0.01382*  &    -0.04444***\\
                    &   (0.00796)   &   (0.00800)   &   (0.01553)   \\
log household meters&    -0.92839   &    -1.31922   &    -0.29035   \\
                    &   (0.85359)   &   (0.92132)   &   (1.53637)   \\
Temperature         &    -0.00332***&    -0.00405***&    -0.00395***\\
                    &   (0.00058)   &   (0.00073)   &   (0.00133)   \\
Temperature squared &     0.00002   &     0.00004   &    -0.00000   \\
                    &   (0.00002)   &   (0.00003)   &   (0.00005)   \\
Daytime             &    -0.04708***&    -0.04502***&    -0.02614   \\
                    &   (0.01018)   &   (0.01084)   &   (0.01884)   \\
Time variables      &         Yes   &         Yes   &         Yes   \\
\midrule
Constant            &        26.6   &        32.2   &        13.3   \\
Observations        &       3,288   &       2,205   &       1,083   \\
\bottomrule

    \end{tabular}
    \text{Robust standard errors are in parentheses. * p<0.10, ** p<0.05, *** p<0.01.}
    \text{Log spot price is instrumented for by wind power prognosis for the same and the other region.}
  \label{tab:hh_17-19}
  \vspace{-0.0cm}
\end{table}


\subsection{The validity of instruments}
\label{subsec:r_validity}
Test for endogeneity (Appendix \ref{app:endogeneity}) and overidentifying restrictions (Appendix \ref{app:overidentifying}).
\medskip\\


\subsection{Heterogeneity and robustness}
\label{subsec:r_robustness}


Heterogeneous effects
\begin{figure}[H]
  \centering
  \caption{Wholesale elasticity by hour}
  \label{fig:ws_elasticity_hour}
\end{figure}

\begin{figure}[H]
  \centering
  \caption{Wholesale peak-elasticity by log grid size}
  \label{fig:ws_elasticity_grid}
\end{figure}


\subsection{Discussion}
\label{subsec:r_discussion}

\subsection{Possible extensions}
\label{subsec:r_extensions}
One possible extension could be to include grid-specific effects other than the random constant term. This is feasible yet cumbersome. The motivation being that one can expect a exists great variation between companies in terms of size, distribution of customers (residential and commercial), industry-intensity. Both at a certain point in time and regarding the time patterns. In our estimation results the time trend does not carry much explanatory power given the other controls, however, this can be due to effects of different direction for different grids covering different areas of the country.
\medskip\\
A very tractable extension would be to use micro data which would allow to control for compositional changes in the presence of heterogeneous consumers.
