\label{sec:data}
% Main characteristics of the data set: source, type of data
% Description of variables used for the	analysis and correspondence with the (ideal) magnitudes in the empirical specification
% Descriptive statistics of the	main variables in the analysis
\subsection{Consumption data}
\label{subsec:data_consumption}
The Danish Transmission System Operator (TSO), Energinet\footnote{\url{https://en.energinet.dk}}, provide hourly aggregated consumption data since January 2016 for each of the 68 grid companies disaggregated into hourly-settled consumption, flex settled consumption and residual consumption. This allows us to distinguish between commercial and residential consumption. Hourly-settled consumption consists of all firms with an annual electricity consumption of at least 100,000 kWh. Flex-settled consumption was introduces in December 2017 such that household consumers can have their electricity consumption settled flexibly according to hour-by-hour prices as well. Though installation and telemeters and flexible settling is only being introduced gradually, this allows a portion of residential consumers and small firms to respond to price changes at an hourly instead of on an annual basis. The residual consumption is the remaining electricity consumption for which there is not remote metering and thus hourly-settling is not a possibility, thus including all households and small firms till December 1 2017.
\par
For each grid we include the number of metering points by category. This data is monthly and



\subsection{Spot market prices}
\label{subsec:data_spot}
For  indicated by the relevant area spot price of the day-ahead-market. Some firms are


\subsection{Tariffs}
From December 2017 grid companies have been allowed to introduce time-of-use (TOU) tariffs in order to send signals that can direct the more flexible tasks away from peak hours. Thus, two of the bigger grid companies have already introduced TIU tariffs for the peak hours from 5-8 PM.


\subsection{Weather data}


The outside temperature is relevant to the extent that electrical heaters or air conditioning is used \citep{lijesen2007real, vesterberg2014residential}.


\subsection{Descriptive statistics}
\label{subsec:variables}

\begin{table}[H]
  \centering
  \caption{Descriptive statistics}
  \footnotesize
    \begin{tabular}{l*{1}{ccccc}}
\hline\hline
                    &\multicolumn{5}{c}{}                                            \\
                    &        mean&          sd&         min&         p50&         max\\
\midrule
Wholesale electricity consumption, MWh&    34.17217&     83.3248&     .063157&      5.9281&    757.5571\\
Retail electricity consumption, MWh&    28.74596&    74.08786&           0&    5.833275&    906.3964\\
Number of wholesale meters&    923.9638&    2440.423&           7&       140.5&       17674\\
Number of retail meters&    57551.23&      148260&         858&       14371&     1006061\\
- of which flex-settled&    4299.473&    37193.14&           0&           0&      596267\\
- of which residual &    53251.75&    134556.1&         855&       14357&      998864\\
Electricity spot price, DKK&    252.9863&    108.0154&     -398.61&      234.94&      1898.9\\
Wind power prognosis for DK1, GWh&    1.225359&    .9221094&           0&       1.002&       3.973\\
Wind power prognosis for DK2, GWh&    .3266759&    .2702798&           0&        .249&       1.084\\
wp                  &    1.075578&    .9126432&           0&        .796&       3.973\\
wp\_other            &    .4764563&    .5610359&           0&         .31&       3.973\\
Wind power prognosis for Sweden, GWh&    1.862874&    1.118507&        .062&       1.668&        5.84\\
Price region DK1 (Western Denmark)&    .8333333&    .3726781&           0&           1&           1\\
Share time-of-use tariff&    .0001378&    .0079703&           0&           0&    .5926748\\
Temperature         &    9.094615&    6.915881&       -11.9&         8.7&        31.4\\
Daytime             &    .5135309&    .4857462&           0&    .6666667&           1\\
Time trend          &    547.4559&    316.4119&           0&         547&        1095\\
Holiday (not in a weekend)&    .0437643&    .2045702&           0&           0&           1\\
\midrule
Observations        &     1420200&            &            &            &            \\
\bottomrule\end{tabular}

  \end{tabular}
  \label{tab:descriptive}
\end{table}
