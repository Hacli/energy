\label{sec:data}
% Main characteristics of the data set: source, type of data
% Description of variables used for the	analysis and correspondence with the (ideal) magnitudes in the empirical specification
% Descriptive statistics of the	main variables in the analysis
The descriptive statistics for the main variables are shown in Table \ref{tab:descriptive} below.

\subsection{Consumption data}
\label{subsec:d_consumption}
The Danish Transmission System Operator (TSO), Energinet\footnote{\url{https://en.energinet.dk}}, provide hourly aggregated consumption data since January 2016 for each of the 52 grid companies disaggregated into hourly-settled consumption, flex settled consumption and residual consumption. This allows us to distinguish between commercial and residential consumption. Hourly-settled consumption consists of all firms with an annual electricity consumption of at least 100,000 kWh. Flex-settled consumption was introduced in January 2018 such that households and small firms can also have their electricity consumption settled flexibly according to hour-by-hour prices. Though installation of automatic meters and flex-settling is only being introduced gradually, this allows a portion of residential consumers and small firms to respond to price changes at an hourly instead of on a quarterly basis. The residual consumption is the remaining electricity consumption for which there is not remote metering and thus hourly-settling is not a possibility, thus including all households and small firms till December 1 2017 and most
\medskip\\
For each grid we include the number of metering points by category. This data is monthly by the \nth{1} of the month. For state-level data it is also common to control for size \citep{burke2017price}.
\begin{figure}[H]
  \centering
  \caption{Electricity consumption by hour (business days)}
  \label{fig:cons_hour}
\end{figure}


\subsection{Spot market prices and wind power prognosis}
\label{subsec:d_spot}
We include the hour-by-hour spot market price on the day-ahead-market for the price region DK1 or DK2 depending on where the grid company is located. As the spot price is negative for a few instances the $log spot price$ is instead set to 0 when the price is less than 1 DKK.
\begin{figure}[H]
  \centering
  \caption{Time series for spot price and total consumption (business days)}
  \label{fig:price_cons_time_series}
\end{figure}
An important factor for the spot price on the day-ahead-market is the hour-by-hour wind power prognosis for the following day.
\begin{figure}[H]
  \centering
  \caption{Wind power prognosis and spot price by hour (business days)}
  \label{fig:wp_price_hour}
\end{figure}


\begin{table}[H]
  \centering
  \caption{Correlations for consumption, spot price, and wind power prognosis}
  \footnotesize
  \label{tab:correlations}
\end{table}


\subsection{Time-of-use tariff}
\label{subsec:d_tout}
From December 2017 grid companies have been allowed to introduce time-of-use (TOU) tariffs in order to send signals that can direct the more flexible tasks away from peak hours. Thus, two of the bigger grid companies have already introduced TIU tariffs for the peak-hours 5-7 PM (from here written as hours 17-19).

\subsection{Weather data}
\label{subsec:d_weather}
The outside temperature is relevant to the extent that electrical heaters or air conditioning is used \citep{lijesen2007real, vesterberg2014residential}. As the electricity consumption ceteris paribus is expected to be higher for both low-end and high-end temperatures, $temperature^2$ is included as well.
\medskip\\
An indicator for daytime is included as well. Such that $daytime=1$ for hours between sunrise and sunset and e.g. $daytime=0.25$ for $hour=7$ if sunrise was a quarter past 7.

\subsection{Time variables}
\label{subsec:d_time}
Year dummies as well as a time trend indicating the number of days since January 1 2016 are included to account for economic growth (increase), technological progress (decrease) \citep{lijesen2007real}, or compositional changes that can affect electricity consumption other than the number of meters.
\medskip\\
Danish bank holidays as well as a few other common holidays are taken into account in order to do sample split regressions for business days and non-business days, the latter including holidays and weekends.
\begin{table}[H]
  \centering
  \caption{Descriptive statistics}
  \footnotesize
    \begin{tabular}{l*{1}{ccccc}}
\hline\hline
                    &\multicolumn{5}{c}{}                                            \\
                    &        mean&          sd&         min&         p50&         max\\
\midrule
Wholesale electricity consumption, MWh&    34.17217&     83.3248&     .063157&      5.9281&    757.5571\\
Retail electricity consumption, MWh&    28.74596&    74.08786&           0&    5.833275&    906.3964\\
Number of wholesale meters&    923.9638&    2440.423&           7&       140.5&       17674\\
Number of retail meters&    57551.23&      148260&         858&       14371&     1006061\\
- of which flex-settled&    4299.473&    37193.14&           0&           0&      596267\\
- of which residual &    53251.75&    134556.1&         855&       14357&      998864\\
Electricity spot price, DKK&    252.9863&    108.0154&     -398.61&      234.94&      1898.9\\
Wind power prognosis for DK1, GWh&    1.225359&    .9221094&           0&       1.002&       3.973\\
Wind power prognosis for DK2, GWh&    .3266759&    .2702798&           0&        .249&       1.084\\
wp                  &    1.075578&    .9126432&           0&        .796&       3.973\\
wp\_other            &    .4764563&    .5610359&           0&         .31&       3.973\\
Wind power prognosis for Sweden, GWh&    1.862874&    1.118507&        .062&       1.668&        5.84\\
Price region DK1 (Western Denmark)&    .8333333&    .3726781&           0&           1&           1\\
Share time-of-use tariff&    .0001378&    .0079703&           0&           0&    .5926748\\
Temperature         &    9.094615&    6.915881&       -11.9&         8.7&        31.4\\
Daytime             &    .5135309&    .4857462&           0&    .6666667&           1\\
Time trend          &    547.4559&    316.4119&           0&         547&        1095\\
Holiday (not in a weekend)&    .0437643&    .2045702&           0&           0&           1\\
\midrule
Observations        &     1420200&            &            &            &            \\
\bottomrule\end{tabular}

  \end{tabular}
  \label{tab:descriptive}
\end{table}
