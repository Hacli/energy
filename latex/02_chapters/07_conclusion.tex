\label{sec:conclusion}
\begin{comment}
1. Introduktion -> hvorfor spændende; ændret marked 
(2. Tidligere studier -> kobles i stedet for unde resultater) 
3. Teori -> Udbud og efterspørgselskarakteristika 
4. Data ? 
5. Estimation
6. Resultater og diskussion: Måske i kombination med andre initiativer, højere grad af eksponering til de faktiske priser, men stadig meget, der kan undersøges endnu 
\end{comment}
The electricity market is and has been undergoing great changes. The energy market and related policy areas are the subject of increasing political attention, in response to the climate crisis. If the electricity sector is to become zero-emission it is unlikely to happen without inducing changes in energy consumption behaviours. From a theoretical point of view there are many reasons to expect inelastic demand; there are few and no close substitutes, demand is indirect and it is deemed a necessary good. Still it constitutes a relatively big expense and prices can grow very high when production is low and demand is high so some response is still to be expected. We estimate hour-by-hour price elasticities for wholesale and retail consumers in Denmark to evaluate the prospects of using price tools to induce such adaptions in the consumption. 
\bigskip

The hour-by-hour price elasticities of electricity consumption are estimated using hourly grid-area-level consumption data stretching from January 2016 to December, 2018. In order to obtain valid estimates we apply Random Effects Instrument Variable-estimation. To overcome the problems of endogeneity from a regression of quantity and price that are jointly determined electricity prices are instrumented. We use wind-power production as an instrument. Wind-power is unlikely to affect electricity consumption through other channels than via the price. Due to merit order effects and because wind-power constitutes a significant share of Danish energy production this has great impact on electricity prices. 
\bigskip 

We obtain estimates that range between -0.019 and -0.048 for wholesale consumers, while the price responsiveness of retail consumers span from no response (zero-estimate) to -0.035 for retail consumers.  While consumption overall is inelastic it holds that wholesale consumption more price-responsive than retail consumption which is in line with our theoretical predictions. Additionally whate underlies thesIn addition to this it is important to note 

Assumption of constant elasticities => may not hold in reality. 

We also conduct a separate analysis for the grid-area covered by Radius where a time-of-use tariffs has been introduced in 2018. This allows for a "first-look"  and evaluate whether this had an effect. 

\begin{comment}
Kopi af abstract: 
  Our results are estimated using the Random Effects Instrument Variable-estimator (REIV). Electricity prices are estimated using wind-power production to overcome concerns of endogeneity. Our estimated price elasticities of -0.045 and -0.027 (for wholesale vis-a-vis retail consumers) are small in size. Similarly we find a relatively small effect from the introduction of a time-of-use tariff. Overall results suggest a limited scope for decentralized, price-based tools to alter electricity demand, but these are not conclusive and further research should be devoted to this. 
  
\end{comment}
\begin{comment}
Gammel konklusion: 
We estimate statistically significant own-price elasticities of demand for wholesale consumers and a statistically significant effect of the time-of-use (TOU) tariff in the grid company Radius. However, the economic magnitude of the quite modest is debatable.
\medskip \\
Literature in this field is quite large but there is still substantial room for improvement, especially within the field of estimating hour-by-hour responses at the micro-level to capture heterogeneity in this aspect.
\end{comment}