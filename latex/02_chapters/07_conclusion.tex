\label{sec:conclusion}
We estimate statistically significant own-price elasticities of demand for wholesale consumers and a statistically significant effect of the time-of-use (TOU) tariff in the grid company Radius. However, the economic magnitude of the quite modest is debatable.
\medskip \\
Literature in this field is quite large but there is still substantial room for improvement, especially within the field of estimating hour-by-hour responses at the micro-level to capture heterogeneity in this aspect.

Kopi af abstract: 
We estimate the hour-by-hour price elasticity of electricity consumption for wholesale and retail consumers in Denmark using hourly grid-level consumption data from January, 2016 to December, 2018.  Our results are estimated using the Random Effects Instrument Variable-estimator (REIV). Electricity prices are estimated using wind-power production to overcome concerns of endogeneity. Our estimated price elasticities of -0.045 and -0.027 (for wholesale vis-a-vis retail consumers) are small in size. Similarly we find a relatively small effect from the introduction of a time-of-use tariff. Overall results suggest a limited scope for decentralized, price-based tools to alter electricity demand, but these are not conclusive and further research should be devoted to this. 