\label{sec:conclusion}
The electricity market is and has been undergoing great changes. The energy market and related policy areas are the subject of increasing political attention, in response to the climate crisis. If the electricity sector is to become zero-emission it is unlikely to happen without inducing changes in energy consumption behaviours. From a theoretical point of view there are many reasons to expect inelastic demand; there are few and no close substitutes, demand is indirect and it is deemed a necessary good. Still it constitutes a relatively big expense and prices can grow very high when production is low and demand is high so some response is still to be expected. We estimate hour-by-hour price elasticities for wholesale and retail consumers in Denmark to evaluate the prospects of using price tools to induce such adaptions in the consumption. \bigskip \par
The hour-by-hour price elasticities of electricity consumption are estimated using hourly grid-area-level consumption data stretching from January 2016 to December, 2018. In order to obtain valid estimates we apply Random Effects Instrument Variable-estimation. To overcome the problems of endogeneity from a regression of quantity and price, that are jointly determined, electricity prices are instrumented. We use wind-power production as our instrument. Wind-power is unlikely to affect electricity consumption through other channels than via the price. Due to merit order effects and because wind-power constitutes a significant share of Danish energy production this has great impact on electricity prices.\bigskip \par \vspace{-1.5mm}
We obtain estimates that range between -0.019 and -0.048 for wholesale consumers, while the price responsiveness of retail consumers span from no response (zero-estimate) to -0.035 for retail consumers.  All in all consumption is largely inelastic and it holds that wholesale consumption is more price-responsive than retail consumption. This suggests that responsiveness is likely related to the degree of exposure to the real-time electricity price fluctuations. A separate analysis for the grid-company Radius,  where a time-of-use tariffs was introduced in late 2017, is conducted. Here we find a small, but economically insignificant reduction in electricity consumption from the tariff. This "first-inspection" of the tariff suggests that it may have been poorly implemented and/or that it is a poor policy tool. Our analysis point in the direction of more centralized solutions as an optimal approach to establish a more sustainable energy production and consumption. 
\begin{comment}

Literature in this field is quite large but there is still substantial room for improvement, especially within the field of estimating hour-by-hour responses at the micro-level to capture heterogeneity in this aspect.
\end{comment}