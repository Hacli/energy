\label{sec:theory}
% Theoretical arguments in the literature closely related to your study

\subsection{The electricity market}
\label{subsec:t_market}
% In Denmark the model assimilates that of a fully liberalized market with retail competition . <- Where to include this?? 
In order to understand how the price of electricity is formed it is necessary understand how the nature of the supply side, demand side and the workings of the electricity market. The electricity market differs from the majority of other markets because demand and supply must synchronize completely at all times. Storing electricity is possible, but at best highly inefficient and thus too costly to implement practically. Instead supply must be at least as great as demand at all times if blackouts are to be avoided. Historically this has been ensured through the production of a surplus of electricity. This is, however, costly and associated with a  negative externality due to carbon dioxide emissions from power plants. \smallskip\\
The electricity market consists of both the physical infrastructure required for electricity generation and transport while it on the other hand is also a well-organised market.  
\smallskip\\
There are several ways to organise the market. In the European Union most of the decisions related to the organisation of the energy market happen at supra-national level. In recent years the electricity market has undergone great changes following the Third Energy Package. The package aims at improving the functioning of the energy market by ensuring more competition and transparency through unbundling of suppliers from operators, greater independence of regulators, more cross-border cooperation and better transparency in retail markets. \footnote{\url{https://ec.europa.eu/energy/en/topics/markets-and-consumers/market-legislation}}. This has increased the number of actors on the electricity market that now comprises producers, distributors, TSO, DSO, balance responsible. 
Actors on the market 
This improve conditions for the consumers while also has perspective of climate. 
decarbonise 
"Clean Energy for all Europeans" package. 

% Also write something about: Priority given to renawable sources. 
In recent years the electricity market has undergone many changes to induce competition and reduce surplus production and thus "unnecessary" carbon dioxide emissions. The move towards more market liberalization still recognizes that that the distribution net constitutes a natural monopoly. In many countries including Denmark the firm in charge maintaining and building the grid is still state-controlled while the remaining market operators are private. Competition is then ensured by letting thirds parties get access to the electric grids in a transparent way. This is only one of several ways to organize a market which has been adopted by the EU, that furthermore wants to promote a single energy market. This single market  implies linking the electrical grids throughout Europe such that the electricity price is the same across all of Europe. As of now Nord Pool has already combined 13 markets; the Nordic countries, the Baltic countries, Germany, Austria, France, Belgium, the Netherlands and the UK, now constitute a single electricity market. The price still varies across countries due to limited cross border transfer capacity. For instance electricity consumers in Denmark also face different prices depending on what side of The great Belt they are situated. \footnote{\url{https://www.nordpoolgroup.com/the-power-market/Integrated-Europe/}} \smallskip \\

Firms and residential households enter the energy market differently.They face different prices that are formed in different ways. This is described further below.  

\subsubsection{The retail market} 
\label{subsubsec: t_resmarket}
%Aktørerne: Slutkunder, elleverandører (distributors) og netvirksomheder. 
The retial market is comprised by the suppliers and the consumers. The group of consumers consists of both small firms and residential consumers. In the retail-market the suppliers act as intermediaries between the power generators and the consumers. The suppliers then enter the wholesale market (described below) on behalf of the retail consumers and buy electricity from the generators. The consumers are then offered a contract that typically implies that the retail consumers face a single fixed price.\smallskip \\

The residential electricity consumers have historically not been treated as 'genuine demanders' \citep{kirschen2003demand}. Instead of facing the actual cost of electricity they instead sign contracts where a distributor acts as a middleman that trades electricity in the market on behalf of its customers, but they receive a premium for undertaking the market risk. The distributors and the consumers then undergo contracts where the price of electricity is typically fixed for up to a year. This insulates the retail consumers from the spot price that better reflects the cost of electricity production at a given point in time. This "distance" to the actual price of electricity is exacerbated even more by the tariffs on electricity that the residential consumers face. The electricity tariffs are particularly high in Denmark where they make up around 68 percent of the price paid for electricity. \footnote{\url{https://www.dr.dk/nyheder/penge/skatter-og-afgifter-aeder-din-elregning-0}} \smallskip \\ 

It is, however, worth noting that the way residential customers are settled is likely to change in the future as smart meters are adopted more widely. For instance Denmark has decided to enrol smart meters to all consumers by 2020 in the Energy Agreement from 2013. In Denmark several grid companies have already rolled out the smart meters which allows for more flexible settlement such that demand can respond to different prices. 

\subsubsection{The wholesale market}
\label{subsubsec: t_whomarket}
Large scale electricity consumers \footnote{In Denmark this entail firms that consume more than 100.000 kWh  a year, to whom hourly settlement is obligatory} enter into the wholesale market for electricity. Here electricity is bought and sold in different markets depending on how well in advance before the actual time of delivery the electricity is traded. Electricity is thus traded via 

\begin{description}

    \item [Long term contracts] 
    Electricity bought and sold further ahead of time than the day before consumption can be agreed upon by undergoing long term contracts or from trades in the forward market. In the forward market futures, forwards, Electricity Price Area Differentials (EPADs) and put and call options are traded. The products are traded either bilaterally or as stocks at NASDAQ OMX Commodities and serves as a way to reduce risks by ensuring a fixed price or insurance against realized price differentials. The value of the futures (and forwards) shifts based on the reference price that is the official nordic day-ahead price. 
    
    \item [The day-ahead market (The spot-market)]
    The day-ahead market is where the majority of electricity is traded either for specific hours or blocks thereof. The price is determined in an auction where all bids and asks are aggregated to form the hourly supply and demand - while the market clearing price is determined by where they intersect subject to the capacity constraints in the market. All the actors in the market (generators, distributors and wholesale clients) pay or receive the same price within a price region. Distributional bottlenecks between regions entails price differences within the market. This price, also referred to as the spot price, thus reflects how much power producers believe they can supply which in turn depends on weather prognoses, expected plant shutdowns etc. but also how much consumers (retail and wholesale) are expected to consume given the physical constraints of the electric grid. It should be noted that Nord Pool Spot have both lower and upper price caps outside of which bids are reduced by a fixed percentage rate.   
  
    \item [The intra-day market]
     The day-ahead market closes at 12 pm the preceding day but from 2 pm and up until an hour before time of delivery trade can occur on the intra-day market where. Here electricity is sold in blocks, hours and 15 minute intervals. Similar to the spot-market this is operated by Nord Pool. The quantities traded in the intra-day market is much smaller than the day-ahead market but this is likely to change as a larger share of the production capacity is constituted by renewables. %Pay-as-bid  The position is open when actual demand and/ or supply differ from the contracted quantities.
    
    \item [The balancing market]
    If gaps between supply and demand reamin after the closing of the intra-day market they must be balanced by the responsible system operator. Each of the actors in the market rarely live completely up to their obligations for instance more or less wind power can be produced or firms may consume unforeseeable large amounts of electricity. This necessitates that the responsible system operator (the TSO, Energinet in Denmark) balances during the delivery period. KILDE: Bog.  Convergence to the expected price => not necessary to account for. 
    
\end{description}

 %Price peaks are necessary for the wholesale market to invest in additional/reserve capacity. 
The power prices are quite volatile which is a result of the low elasticities for both supply and demand in the short run.

\textbf{Production of electricity}
Electricity is supplied by different kinds of power plants that each has their own advantages in terms of when they can produce and how fast they reach an efficient production level. In most of Europe the most common types of power include lignite, coal, gas, nuclear, solar, wind and hydro, ranged from the more emission intense to the least. The marginal cost of electricity from renewable sources are far because almost no (costly) inputs are needed, but these typically require certain weather conditions outside the control of the supplier. Coal and lignite power plants have higher costs running and have to be running for a while to reach efficient production, but do not rely on external parameters. These are typically used as a part of the base-load  and are typically used as part of the base-load. The available production capacities and weather conditions thus shape the supply curve. \smallskip \\

For each supplier it is optimal (at least in the short term) to ask for the marginal cost of producing electricity at a given point in time. This implies that the supply curve and thus the order in which generators are dispatched reflects the merit order. If the weather conditions are right electricity from renewable sources are dispatched first and then hard coal and lignite plants. If the share of renewables in the productional capacity is large the supply curve is shifted to the right. The marginal costs of electricity production from carbonizing plants are even higher in the EU where the supplier has to buy emission quotas such that renewable and low-emission power production is prioritized. \smallskip \\

All receive the same price such that low cost production rewarded. Convergence of prices would be obtained if the interconnection capacity is sufficient.

In the case where the demand for electricity is particularly high i.e. at a peak costs (financial and external) very high are also very high and mainly high-emission. Demand peak => higher prices that are sufficiently high to cover costs of higher cost production units 
when answering to spikes in demand.  

Cost of capacity is charged as a flat transmission grid fee to all consumers. 

*Negative prices. Renewables and must-run capacities may in combination produce a greater supply of power than what is demanded in which case the price turns negative. This reflects that there a cost associated to running the plant that producers will want to cover to any extent possible even if this means a. 
For our analysis we aim at estimating the implicit own-price elasticity of demand by assuming the individual firms takes the spot price as given and plans it's hourly electricity use accordingly.
\smallskip \\

%Domestic firms with an annual electricity consumption of at least 100,000 kWh in general have their hourly consumption charged by the corresponding spot market price. Between December 2017 and 2020 it is gradually phased in that. Smaller firms and households likewise have their hourly electricity consumption settled at the spot market price.

\subsection{Theory of demand-side response to electricity prices}
\label{subsec:t_demand}
The market of electricity itself differs from most other markets as described above. This complexity of the market and how the price is shaped is even higher because the nature of the demand for electricity is of indirect character. Electricity demand, for retail and wholesale consumers alike, is shaped by the demand for the use of other appliances that require electricity to function. 
This indirectness implies that even less information on consumption is available to the consumer which makes reactions difficult. Energy consumption is thus typically decided upon according to behavioural rules. \citep{kirschen2003demand}. 

High consumer surplus -> electricity is difficult to replace (no close substitutes) 

\begin{align}
  \begin{split}
  \ln e_{i,t}= &\beta_0 + \beta_1 \ln E_{i,t-1} + \beta_2 P_{it} 
  \end{split}
\end{align}


In terms of demand of electricity there is an important distinction to make between residential and wholesale electricity consumers. They differ in how flexible they are and how they pay for electricity which makes , 

For wholesale consumers electricity typically also constitute a much greater share of all costs as compared to residential consumers. 
On the one hand wholesale are subject to more restrictions in terms of when they operate as this is typically in business hours depending, of course, on the nature of the firm i.e. what industry it is in, how energy intensive consumption is, not very much in service sector, what peak hours are etc. Also different to what extent it is possible to postpone production, alter schedule osv. On the other hand consumption constitutes a great share of costs such that incentive to improve energy efficiency. More flexibility because the prices that they face are settled in the wholesale market to which they have direct access. Prices are typically close to the spot price such that firm more or less knows one day ahead the electricity prices it faces the next coming day and plan accordingly. Knowing prices allow them to much greater extend respond to changes. 

The residential consumers on the other hand. 
Two obvious limitations to the flexibility of demand is that demand responses require knowledge of prices and that one can assume taking action in order to reduce demand at a certain hour carries a fixed cost of e.g. \$5 which makes is optimal for small consumers to ignore spot market except at extreme prices or when time-of-use (TOU) tariffs are introduced \citep{wolak2011residential}. 

This all adds up to a weak elasticity. Another aspect has to do with how electricity is regarded as a good. \citep{kirschen2003demand} points to the fact that electricity is regarded as a good that is indispensable and essential to quality of life. It has always been marketed as easily accessible in terms of usage and availability although this may not genuinely be the case. In addition all consumption of electricity is indirect and consumption happens through the usage of other goods such as electrical appliances, entertainment etc. Distinction between some consumption that is foregone - lightning, listening to music, the radio etc. while other is postponed (dryer, washing machine etc.) 


%- Kommentér på kort og lang sigtet efterspørgsel. Forskellen imellem disse. In the short term capital cannot be adjusted to the optimum. -> Motivates using a partial adjustment model  
It is likely to get easier to smooth load profiles in the future as smart meters become more common, but as of now they are not rolled out yet. This is a necessity for interruptable contracts. 

This does not allow price spikes to form which may not be a good thing because they highlight what state the market is in - so it is unclear to the producers. 

Something about tariffs. They are large may overshadow any price fluctuations and thus make it more difficult for residential consumers to adjust their consumption. 

\citep{kirschen2003demand}

%\subsection{Simultaneity of demand and supply}
%\label{subsec:t_simultaneity}
%Our data set contains all firms that have their hourly consumption charged by the corresponding spot market price, however, for big companies options exist to negotiate a less volatile settling or to buy financial products in order to increase price security. Our data does not cover firms that buy their electricity directly on the spot market or bilaterally by the electricity supplier on the so-called over-the-counter (OTC) market. It is a clear advantage that we know all of the demand in our firm data is directly subject to the spot price \citep{lijesen2007real}.



