\label{sec:theory}
% Theoretical arguments in the literature closely related to your study

\subsection{The electricity market}
\label{subsec:t_market}
In order to understand how the price of electricity is formed it is necessary understand how electricity is produced and sold on the one hand and how it is bough and demanded on the other. The electricity market differs from the majority of other markets because demand and supply must synchronize completely at all times. Storing electricity is possible, but at best highly inefficient and thus too costly to implement practically. Instead supply must be at least as great as demand at all times if blackouts are to be avoided. Historically this has been ensured through the production of a surplus of electricity. \smallskip\\

Because demand demand typically occurs this demands large production capacities that can start production when this is needed. Because plants are started in the merit order these peaks implies that production is extra costly (also in terms of carbon dioxide emissions) when answering to these spikes in demand. 
\smallskip\\

In recent years the electricity market has undergone many changes to induce competition and reduce surplus production and thus "unnecessary" carbon dioxide emissions. The move towards more market liberalization still recognizes that that the distribution net constitutes a natural monopoly. In many countries including Denmark the firm in charge maintaining and building the grid is still state-controlled while the remaining market operators are private. Competition is then ensured by letting thirds parties get access to the electric grids in a transparent way. This is only one of several ways to organize a market which has been adopted by the EU, that furthermore wants to promote a single energy market. This single market  implies linking the electrical grids throughout Europe such that the electricity price is the same across all of Europe. As of now Nordpool has already combined 13 markets; the Nordic countries, the Baltic countries, Germany, Austria, France, Belgium, the Netherlands and the UK, now constitute a single electricity market. The price still varies across countries due to limited cross border transfer capacity. For instance electricity consumers in Denmark also face different prices depending on what side of The great Belt they are situated. \cite{https://www.nordpoolgroup.com/the-power-market/Integrated-Europe/} \smallskip \\
Firms and residential households enter the energy market differently.They face different prices that are formed in different ways. This is described further below.  

\subsubsection{The wholesale market}
\label{subsubsec: t_whomarket}

Electricity is also traded in the financial market in terms of forwards and futures. Price peaks are necessary for the wholesale market to invest in additional/reserve capacity. 

 

The final price paid by the firm is .. 
Bids in the day-ahead-market forms the hourly price\footnote{\url{nordpoolgroup.com/trading/Day-ahead-trading/Price-calculation}} to which individual firms are charged. 

For our analysis we aim at estimating the implicit own-price elasticity of demand by assuming the individual firms takes the spot price as given and plans it's hourly electricity use accordingly.
\medskip\\
Domestic firms with an annual electricity consumption of at least 100,000 kWh in general have their hourly consumption charged by the corresponding spot market price. Between December 2017 and 2020 it is gradually phased in that
smaller firms and households likewise have their hourly electricity consumption settled at the spot market price.
*Smart meters
There are plenty of ways to organize the market. In Denmark the model assimilates that of a fully liberalized market with retail competition, which requires as an independent balancing mechanism. 

\subsubsection{The day-ahead market}
The day-ahead market of electricity is the spot market. This is made up of several country-specific markets . In the spot market electricity for specific hours the next day and blocks thereof are sold in an auction. In the auction all bids and asks are aggregated to form the hourly supply and demand - while the market clearing price is determined by where they intersect. All actors pay or receive the same market clearing price. The majority of electricity is traded in the day-ahead market. %How is this linked to the intra-day market
The day-ahead market closes at 2 pm the preceding day. All additional electricity trade takes place in the intra-day market where electricity is sold in blocks, hours and 15 minute intervals. The position is open when actual demand and/ or supply differ from the contracted quantities. If there are gaps after closing of the intra-day market they must be balanced by the responsible system operator. \todo{The prices of balancing energy is the average price of control energy and pay-as-bid.} 



In the market foreign suppliers are limited due to constraints from inter-connector capacities across country borders. This implies that prices are heterogeneous across the different local markets in the Nordpool spot market. 

Bullets to write about: 
* Day-ahead markets
* Hour ahead markets 
Also remember to put references! 

Potential problems: 


\subsection{Simultaneity of demand and supply}
\label{subsec:t_simultaneity}
Our data set contains all firms that have their hourly consumption charged by the corresponding spot market price, however, for big companies options exist to negotiate a less volatile settling or to buy financial products in order to increase price security. Our data does not cover firms that buy their electricity directly on the spot market or bilaterally by the electricity supplier on the so-called over-the-counter (OTC) market. It is a clear advantage that we know all of the demand in our firm data is directly subject to the spot price \citep{lijesen2007real}.


\subsubsection{The retail market} %Er det her den rigtige oversættelse? 
\label{subsubsec: t_resmarket}
The residential electricity consumers have historically not been treated as 'genuine demanders' \citep{kirschen2003demand}. Instead of facing the actual cost of electricity they instead sign contracts where a distributor acts as a middleman that trades electricity in the market on behalf of its customers, but they receive a premium for undertaking the market risk. The distributors and the consumers then undergo contracts where the price of electricity is typically fixed for up to a year. This insulates the retail consumers from the spot price that better reflects the cost of electricity production at a given point in time. This "distance" to the actual price of electricity is exacerbated even more by the tariffs on electricty that the residential consumers face. The electricity tariffs are particularly high in Denmark where they make up around 68 percent of the price paid for electricity. \smallskip \\ 

It is, however, worth noting that the way residential customers are settled is likely to change in the future as smart meters are adopted more widely. For instance xx has decided to enrol smart meters to all consumers by 20xx. In Denmark several grid companies, among those Radius, have ... . More on flexible tariffs and demand responses below.  



\subsection{Theory of demand-side response to electricity prices}
\label{subsec:t_demand}
The market of electricity itself differs from most other markets as described above. This complexity of the market and how the price is shaped is even higher because the nature of the demand for electricity is of indirect character. Electricity demand is shaped by the demand for the use of a multitude of appliances that require electricity to function. This indirectness implies that even less information on consumption is available. Energy consumption is thus typically decided upon according to behavioural rules. \citep{kirschen2003demand}. 

In terms of demand of electricity there is an important distinction to make between residential and wholesale electricity consumers. They differ in how flexible they are and how they pay for electricity which makes , 

For wholesale consumers electricity typically also constitute a much greater share of all costs as compared to residential consumers. 
On the one hand wholesale are subject to more restrictions in terms of when they operate as this is typically in business hours depending, of course, on the nature of the firm i.e. what industry it is in, how energy intensive consumption is, not very much in service sector, what peak hours are etc. Also different to what extent it is possible to postpone production, alter schedule osv. On the other hand consumption constitutes a great share of costs such that incentive to improve energy efficiency. More flexibility because the prices that they face are settled in the wholesale market to which they have direct access. Prices are typically close to the spot price such that firm more or less knows one day ahead the electricity prices it faces the next comming day and plan accordingly. Knowing prices allow them to much greater extend respond to changes. 

The residential consumers on the other hand. 
Two obvious limitations to the flexibility of demand is that demand responses require knowledge of prices and that one can assume taking action in order to reduce demand at a certain hour carries a fixed cost of e.g. \$5 which makes is optimal for small consumers to ignore spot market except at extreme prices or when time-of-use (TOU) tariffs are introduced \citep{wolak2011residential}. 

 

This all adds up to a weak elasticity. Another aspect has to do with how electricity is regarded as a good. \citep{kirschen2003demand} points to the fact that electricity is regarded as a good that is indispensable and essential to quality of life. It has always been marketed as easily accessible in terms of usage and availability although this may not genuinely be the case. In addition all consumption of electricity is indirect and consumption happens through the usage of other goods such as electrical appliances, entertainment etc. Distinction between some consumption that is foregone - lightning, listening to music, the radio etc. while other is postponed (dryer, washing machine etc.) 


// NOTER // 
- Måske generelt hvordan ser det ud med efterspørgslen i DK. Lidt under halvdelen - knap 45\%. 
- Household production theory 
- Kommentér på kort og lang sigtet efterspørgsel. Forskellen imellem disse. In the short term capital cannot be adjusted to the optimum. -> Motivates using a partial adjustment model  
- Compared to other countries we may expet that the responses to climate variables are smaller in Denmark. Only few households have electrical heating + use of air condition is not widespread as temperatures in summer does not get that warm. 

It is likely to get easier to smooth load profiles in the future as smart meters become more common, but as of now they are not rolled out yet. This is a necessity for interruptable contracts. 


Many things have been considered. Should we name different demand response mechanisms? 

This does not allow price spikes to form which may not be a good thing because they highlight what state the market is in. 

Something about tariffs. They are large may overshadow any price fluctuations and thus make it more difficult for residential consumers to adjust their consumption. 

\citep{kirschen2003demand}
 



