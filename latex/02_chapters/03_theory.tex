\label{sec:theory}
% Theoretical arguments in the literature closely related to your study

\subsection{The electricity market design}
\label{subsec:t_market}
% In Denmark the model assimilates that of a fully liberalized market with retail competition . <- Where to include this??
In order to understand how the price of electricity is formed it is necessary understand how the nature of the supply side, demand side and the workings of the electricity market. The electricity market differs from the majority of other markets because demand and supply must synchronize completely at all times. Storing electricity is possible, but at best highly inefficient and thus too costly to implement practically. Instead supply must be at least as great as demand at all times if blackouts are to be avoided. Historically this has been ensured through the production of a surplus of electricity. This is, however, costly both in terms of inefficiency and because of the associated negative externalities due to carbon dioxide emissions from fossil fueled power plants.
\medskip

The electricity market consists of both the physical infrastructure required for electricity generation and transport while it on the other hand is also a well-organized market.
\medskip

There are several ways to organize the market. Within the European Union most of the decisions related to the organization of the energy market happen at the supranational level. It In recent years the electricity market has undergone great changes following the Third Energy Package. The package aims at improving the functioning of the energy market by ensuring more competition and transparency through unbundling of suppliers from operators, greater independence of regulators, more cross-border cooperation and better transparency in retail markets. \footnote{\url{https://ec.europa.eu/energy/en/topics/markets-and-consumers/market-legislation}}. This has increased the number of actors on the electricity market that now comprises consumers, producers, distributors, Independent TSO (Transmission System Operators - the owner of the transmission infrastructure), DSO (Distribution System Operators) and balance responsible actors.
\medskip
%Skriv disse to sektioner sammen
% Also write something about: Priority given to renewable sources.
In recent years the electricity market has undergone many changes to induce competition and reduce surplus production and thus "unnecessary" carbon dioxide emissions. The move towards more market liberalization still recognizes that that the distribution net constitutes a natural monopoly. In many countries including Denmark the firm in charge maintaining and building the grid is still state-controlled while the remaining market operators are private. Competition is then ensured by letting thirds parties get access to the electric grids in a transparent way. This is only one of several ways to organize a market which has been adopted by the EU, that furthermore wants to promote a single energy market. This is elaborated in section \ref{subsec:t_EU}.
\par 

Firms and residential households enter the energy market differently. They face different prices that are formed in different ways. This is described further below.

\subsubsection{The retail market}
\label{subsubsec: t_resmarket}
%Aktørerne: Slutkunder, elleverandører (distributors) og netvirksomheder.
The retail market is comprised by the suppliers and the consumers. The group of consumers consists of both small firms and residential consumers. In the retail-market the suppliers act as intermediaries between the power generators and the consumers. The suppliers then enter the wholesale market (described below) on behalf of the retail consumers and buy electricity from the generators. The consumers are then offered a contract that typically implies that the retail consumers face a single fixed price. %Kommentér på at det nok kommer til at ændre sig i fremtiden da EU-lov tilsiger at brugerne skal tilbydes flere forskellige typer af kontrakter.
\smallskip \\
The residential electricity consumers have historically not been treated as 'genuine demanders' \citep{kirschen2003demand}. Instead of facing the actual cost of electricity they instead sign contracts where a distributor acts as a middleman that trades electricity in the market on behalf of its customers, but they receive a premium for undertaking the market risk. The distributors and the consumers then undergo contracts where the price of electricity is typically fixed for up to a year. This insulates the retail consumers from the spot price that better reflects the cost of electricity production at a given point in time. This "distance" to the actual price of electricity is exacerbated even more by the tariffs on electricity that the residential consumers face. These are particularly high in Denmark as they on average make up 62 percent of electricity bills excluding VAT (20 percent).\footnote{\href{https://elpris.dk}{elpris.dk}}. In addition to the cost of electricity itself each consumer pays a distribution grid tariff, a transmission grid tariff, an electricity tax, PSO (Public service obligation) and sales tax (VAT).
\medskip

It is, however, worth noting that the way residential customers are settled is likely to change in the future as smart meters are adopted more widely. \todo{Explain what smart meters are}.
\medskip

For instance Denmark has decided to enrol smart meters to all consumers by 2020 in the Energy Agreement from 2013. In Denmark several grid companies have already rolled out the smart meters which allows for more flexible settlement such that demand can respond to different prices. %What are smart meters?

\subsubsection{The wholesale market}
\label{subsubsec: t_whomarket}
Large scale electricity consumers \footnote{In Denmark this entail firms that consume more than 100.000 kWh  a year, to whom hourly settlement is obligatory.} enter into the wholesale market for electricity. Here electricity is bought and sold in different markets depending on how well in advance before the actual time of delivery the electricity is traded. Electricity is thus traded via

\begin{description}

    \item [Long term contracts]
    Electricity bought and sold further ahead of time than the day before consumption can be agreed upon by undergoing long term contracts or from trades in the forward market. In the forward market futures, forwards, Electricity Price Area Differentials (EPADs) and put and call options are traded. The products are traded either bilaterally or as stocks at NASDAQ OMX Commodities and serves as a way to reduce risks by ensuring a fixed price or insurance against realized price differentials. The value of the futures (and forwards) shifts based on the reference price that is the official nordic day-ahead price.

    \item[The day-ahead market]
    The day-ahead market (The spot-market) is where the majority of electricity is traded either for specific hours or blocks thereof. The price is determined in an auction where all bids and asks are aggregated to form the hourly supply and demand - while the market clearing price is determined by where they intersect subject to the capacity constraints in the market. All the actors in the market (generators, distributors and wholesale clients) pay or receive the same price within a price region. Distributional bottlenecks between regions entails price differences within the market. This price, also referred to as the spot price, thus reflects how much power producers believe they can supply which in turn depends on weather prognoses, expected plant shutdowns etc. but also how much consumers (retail and wholesale) are expected to consume given the physical constraints of the electric grid. It should be noted that Nord Pool Spot have both lower and upper price caps outside of which bids are reduced by a fixed percentage rate.

    \item [The intra-day market]
     The day-ahead market closes at 12 pm the preceding day but from 2 pm and up until an hour before time of delivery trade can occur on the intra-day market where. Here electricity is sold in blocks, hours and 15 minute intervals. Similar to the spot-market this is operated by Nord Pool. The quantities traded in the intra-day market is much smaller than the day-ahead market but this is likely to change as a larger share of the production capacity is constituted by renewables. From 2016 to 2017 the traded volume in the intraday-market increased by 35 per cent for the Nordic/Baltic/German markets.  t\footnote{\url{https://www.nordpoolgroup.com/globalassets/download-center/annual-report/annual-report-nord-pool_2017.pdf}}  %Pay-as-bid  The position is open when actual demand and/ or supply differ from the contracted quantities. or https://www.nordpoolgroup.com/message-center-container/newsroom/exchange-message-list/2019/q2/nord-pool-key-statistics--april-2019/

    \item [The balancing market]
    If gaps between supply and demand reamin after the closing of the intra-day market they must be balanced by the responsible system operator. Each of the actors in the market rarely live completely up to their obligations for instance more or less wind power can be produced or firms may consume unforeseeable large amounts of electricity. This necessitates that the responsible Transmission System Operator (TSO), Energinet\footnote{\url{https://en.energinet.dk}} in Denmark, balances during the delivery period. %KILDE: Bog.  Convergence to the expected price => not necessary to account for.


\end{description}

 %Price peaks are necessary for the wholesale market to invest in additional/reserve capacity.
%The power prices are quite volatile which is a result of the low elasticities for both supply and demand in the short run.

\subsection{Production and supply of electricity}
\label{subsec:t_production}
Electricity is supplied by different kinds of power plants that each has their own advantages in terms of when they can produce and how fast they reach an efficient production level. In most of Europe the most common types of power include lignite, coal, gas, nuclear, solar, wind and hydro, ranged from the more emission intense to the least. The marginal cost of electricity from renewable sources are far because almost no (costly) inputs are needed, but these typically require certain weather conditions outside the control of the supplier. Coal and lignite power plants have higher costs running and have to be running for a while to reach efficient production, but do not rely on external parameters. These are typically used as a part of the base-load  and are typically used as part of the base-load. The available production capacities and weather conditions thus shape the supply curve.
\medskip 

For each supplier it is optimal (at least in the short term) to ask for the marginal cost of producing electricity at a given point in time. This implies that the supply curve and thus the order in which generators are dispatched reflects the merit order. The merit order effect is illustrated below in figure \ref{fig:merit}. If the weather conditions are right electricity from renewable sources are dispatched first - as illustrated by the blue supply curve, because they have marginal costs that are essentially zero, and then hard coal and lignite plants. If the share of renewables in the production capacity is large the supply curve is shifted to the right compared to the case of no renewables ($S_{NoWind}$).
\medskip 
\begin{figure}[H]
    \centering
    \caption{Merit Order Effect}
    \label{fig:merit}
    \center
        \def\svgwidth{0.9\textwidth}
        \import{03_figures/}{merit.pdf_tex}
\end{figure}

\noindent The marginal costs of electricity production from carbonizing plants are even higher in the EU where the supplier has to buy carbon dioxide emission quotas such that renewable and low-emission power production is prioritized. Construction of such plants have also been given priority through subsidy programs. 
\medskip

In the case where the demand for electricity is particularly high i.e. at a peak then costs are also very high. The extra electricity generation is carried out by plants with high marginal costs. At peak the price is high enough to cover the costs from fossil fuels such as hard coal, gas and oil. The last plants to be dispatched are thus all high emission - and has high financial and external costs cf. \cite{zweifel2017energy}. Covering spikes like this will remain problematic even if the share of renewable energy increases significantly. Demand side management may thus be relevant tool if transitioning to low emission electricity production. Most demand response mechanisms are in intended to encourage shifts in consumption and thus reduce peak demand. Section \ref{subsec:t_demand} goes into greater details with demand. High prices does, however, also have some benefits - at least when it is set in a competitive market. It is when prices are high that producers can recover their capacity costs. Similarly the price serves as a signal to consumers of the state of the market. Price policies such as price caps may distort these mechanisms. 

%*Negative prices. Renewables and must-run capacities may in combination produce a greater supply of power than what is demanded in which case the price turns negative. This reflects that there a cost associated to running the plant that producers will want to cover to any extent possible even if this means a.
%For our analysis we aim at estimating the implicit own-price elasticity of demand by assuming the individual firms takes the spot price as given and plans its hourly electricity use accordingly.
%\smallskip \\

%Domestic firms with an annual electricity consumption of at least 100,000 kWh in general have their hourly consumption charged by the corresponding spot market price. Between December 2017 and 2020 it is gradually phased in that. Smaller firms and households likewise have their hourly electricity consumption settled at the spot market price.

\subsubsection{(European) Market Integration}
\label{subsec:t_EU}
As mentioned in previous sections the European Energy Market has and is undergoing great changes in order to achieve what is considered a more efficient and thus well-functioning market. This is believed to be insured by inducing more competition in the energy sector - not just from ownership unbundling but also through freer movement of energy. Freer movement of energy would thus imply that electricity is allowed to flow to where it is needed the most. The end goal is thus a single, free (European) energy market. This is currently constrained by limited cross-grid transmission capacities. Depending on demand and supply on each side then a connecting capacity make up potential bottlenecks that can result in different prices.%grid congestion
This is evident even for the relatively small market of Denmark where two price regions exists because the Great Belt Link has a capacity of 600 MW, which is too little for prices to converge.
\medskip

Even though capacity constraints are still binding improved integration the Danish market is linked to the Northern European market. Trade happens at the electricity exchange 'Nord Pool'. Nord pool has already combined 13 markets; The Nordic countries, the Baltic countries, Germany, Austria, France, Belgium, the Netherlands and the UK.  This and future integration implies that electricity prices in Denmark are under great influence from production conditions abroad. A low hydro reserve in Norway may thus increase Danish electricity prices and sunshine in Germany may vice versa lower prices.
\medskip

Increased market integration means more harmonised prices and thus lower price volatility. Market integration can pose a solution to achieve an increased share of energy produced by intermittent renewable sources without increasing price volatility and uncertainty. Market integration allows for changes in the composition of the production capacities within the electrical grid to something more optimal where they complement each other. Danish wind energy is thus complemented very well by storaged hydro from Norway \citep{ambec2012electricity}.  This may become increasingly important as the EU is moving towards achieving its target of 50 percent renewable energy production in 2030 following the Paris Agreement.

\subsection{Demand of electricity}
\label{subsec:t_demand}
It is often deemed more cost-efficient to reduce demand in periods of peak demand as opposed to dispatching high-merit generation plants. This can be prompted through demand responses. Demand responses can be defined as the resulting deviations from normal electricity consumption patterns in response to changes in the electricity price over time. These are often designed to induce lower consumption when demand is particularly high \citep{albadi2008summary}. Examples include time-of-use tariffs, real time pricing and demand bidding. If implemented effectively they can benefit individuals through lower bills, but also the entire market by deferring or avoided distribution and transmission upgrades, increasing reliability, reducing price volatility and improving the efficiency of the electricity market. For demand responses to provide these benefits consumption should be price sensitive. There are, however, many reasons why demand for electricity can be quite inelastic. If demand, as expected, turns out to be highly inelastic this either points to policies that try to change the demand curve itself or more centralized solutions where for instance the supply side of the market is targeted instead.
\medskip

The market of electricity itself differs from most other markets as described above. This complexity of the market and its price formation the price is even higher because the nature of the demand for electricity is of indirect character. Electricity demand, for retail and wholesale consumers alike, is shaped by the demand for the use of other appliances that require electricity to function.
This indirectness implies that less information on costs is available to the consumer at the time of consumption which makes responding difficult. In order to calculate the price of using an appliance knowledge of both electricity prices and how much each device uses is required. Elasticities thus answer the limits of using decentralized solutions to help the energy market to clear in an environmentally sustainable way. 
\medskip

In terms of demand of electricity there is an important distinction to make between residential and wholesale electricity consumers. But it is a common feature for both types of consumers that electricity is hard to replace. Much of firms' physical capital run on electricity while the same applies for most household appliances needed for at-home production of food etc. Similarly society's reliance on electricity is increasing as more infrastructure is becoming digitized and moving away from fossil fuels. Electricity thus a necessity for economic growth and functioning of developed and industrialized economies. Still wholesale and retail consumers differ not only in how they enter into the market but also in the nature of their demand. The differences described further below affect how flexible consumers are and different elasticities are thus to be expected.

\subsubsection*{Demand for wholesale consumers}
Wholesale consumers enter into the wholesale market themselves, as indicated, by their name. This implies that wholesale consumers face the real time prices. Wholesale consumers thus have incentives to lower their consumption when it is costly to produce electricity and prices thus are high. With the exception of when price caps are binding they can thus respond to the state of the market. Firms are also mostly exempt from paying tariffs on their electricity.
\medskip

For wholesale consumers electricity typically constitutes a great share of all costs. The possible cost reductions that monitoring of prices makes possible are non-negligible. This suggests that most wholesale consumers should be relatively more price sensitive than retail consumers.
\medskip

On the other hand firms are also subject to more restrictions in regards to when they operate as this is typically in business hours depending, of course, on the nature of the firm i.e. what industry it is in. Similarly firms are often required to deliver a product or a service subject to a contract which does not allow firms to postpone production for too long. This is, however, not likely to affect the hour-by-hour elasticity greatly. %Should it be argued why?

\subsubsection*{Demand theory for retail consumers}
Small scale consumers are more insulated from price fluctuations because they enter the energy market only through a middleman and contracts typically involve a fixed price per kWh albeit this may change in the future. New EU directives dictate that more billing options should be made available to retail consumers in the future. One reason why retail consumers are not flexibly settled yet is that the physical infrastructure required (remotely read meters/smart meters) is not widely installed. In fact many demand responses rely on the consumer being able to inform themselves on the real time prices.
\smallskip \\

Even if residential consumers and small firms faced real-time prices they ...
One obvious limitation to the flexibility of demand is that demand responses require knowledge of prices at a given point in time. Furthermore it can be costly to take actions in order to reduce demand. Combined these factors may make it optimal for small consumers to not adhere to market signals except when prices are extremely high \citep{wolak2011residential}, which could be due to the introduction of a price based demand response). This implies that many consumers have to rely on behavioral rules when deciding how much energy to consume cf. \citep{kirschen2003demand}.
\smallskip \\
 Another aspect has to do with how electricity is regarded as a good. \citet{kirschen2003demand} points to the fact that electricity is regarded as a good that is indispensable and essential to quality of life. It has always been marketed as easily accessible in terms of usage and availability although this may not genuinely be the case.
 \smallskip \\

 Residential consumers is more flexible than industrial in at least one aspect: They are better able to forego some consumption. While some consumption is postponed usage of other appliances constitutes foregone consumption instead. Examples include lighting, entertainment related.

Tariffs constitute a considerable share of the electricity bill for residential consumers as described in section \ref{subsec:t_market}. When tariffs are very large they may overshadow any price fluctuations and thus make it more difficult for residential consumers to adjust their consumption. Tariffs and taxes, unless Pigouvian, are distortionary because prices serve as the suppliers' only signal of the true cost of production at a given point in time.

This all adds up to an expected weak elasticity.

%\subsection{Simultaneity of demand and supply}
%\label{subsec:t_simultaneity}
%Our data set contains all firms that have their hourly consumption charged by the corresponding spot market price, however, for big companies options exist to negotiate a less volatile settling or to buy financial products in order to increase price security. Our data does not cover firms that buy their electricity directly on the spot market or bilaterally by the electricity supplier on the so-called over-the-counter (OTC) market. It is a clear advantage that we know all of the demand in our firm data is directly subject to the spot price \citep{lijesen2007real}.
