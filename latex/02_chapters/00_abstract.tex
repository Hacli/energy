\begin{abstract}\noindent \footnotesize
We estimate the hour-by-hour price elasticity of electricity consumption for wholesale and retail consumers in Denmark using hourly grid-level consumption data from January, 2016 to December, 2018.  Our results are estimated using the Random Effects Instrument Variable-estimator (REIV). Electricity prices are estimated using wind-power production to overcome concerns of endogeneity. Our estimated price elasticities of -0.048 and -0.027 (for wholesale vis-a-vis retail consumers) are small in size. Similarly we find a relatively small effect from the introduction of a time-of-use tariff. Overall results suggest a limited scope for decentralized, price-based tools to alter electricity demand, but these are not conclusive and further research should be devoted to this. 
\\
\textbf{Keywords:}  \scriptsize {Electricity consumption}, {Electricity Price Elasticity}, {Renewable energy}, {Demand responses} 
\\ \\
\footnotesize
\textbf{Keystrokes:} xx.xxx which corresponds to xx standard pages. 
\\
\textbf{Individual contributions:} 
\\ Thor Donsby Noe: 2.1, 2.3, 3.1, 3.1.2, 3.2.1, 4.1, 4.3, 4.5, 5.2, 5.4, 6.1, 6.2.1, 6.4, 6.5.1
\\ Cathrine Falbe Pedersen: 2.2, 2.4, 3.1.1, 3.2, 3.3, 4.2, 4.4, 5.1, 5.3, 5.5, 6.2, 6.3, 6.5, 6.6
\end{abstract}
