\begin{abstract}\noindent \footnotesize
We estimate the hour-by-hour price elasticity of electricity consumption for wholesale and retail consumers in Denmark using hourly consumption data for 48 grid areas from January, 2016 to December, 2018.  Our results are estimated using the Random Effects Instrument Variable-estimator (REIV) where electricity spot prices are instrumented using the prognosis for wind-power production to overcome concerns of endogeneity.\footnote{On Github we provide a well-structured overview of Python and Stata code as well as all estimation tables with the complete set of time controls: \href{https://github.com/thornoe/energy}{github.com/thornoe/energy}} For peak-hours our estimated price elasticities of -0.048 and -0.027 (for wholesale vis-a-vis retail consumers) are small in size. Similarly we find a relatively small effect from the introduction of a time-of-use tariff. Overall results suggest a limited scope for decentralized, price-based tools to alter electricity demand, but these are not conclusive and further research should be devoted to this. 
\medskip \\
\scriptsize
\textbf{Keywords:} {Electricity consumption} {\textbullet} {Electricity Price Elasticity} {\textbullet} {Demand responses} {\textbullet} {Wind Power Production} {\textbullet} {Random Effects Instrumental Variables estimation}
\medskip \\
\footnotesize
\textbf{Keystrokes:} 70,500 which corresponds to 29.4 standard pages. 
\\
\textbf{Individual contributions:} 
\\ Thor Donsby Noe: 2.1, 2.3, 3.1, 3.1.2, 3.2.1, 4.1, 4.3, 4.5, 5.2, 5.4, 6.1, 6.2.1, 6.4, 6.5.1
\\ Cathrine Falbe Pedersen: 2.2, 2.4, 3.1.1, 3.2, 3.3, 4.2, 4.4, 5.1, 5.3, 5.5, 6.2, 6.3, 6.5, 6.6
\end{abstract}
