\label{sec:intro}
% Motivation of the study: why you focus on this particular issue
% Hypothesis and objective(s)
% Description of the background (theoretical and empirical) that lead you to propose the hypothesis
% Approach and summary of results: what is your strategy to check the hypothesis and the main result
% Structure of the paper
The focus of this paper is to estimate how the hour-by-hour electricity consumption responds to hourly electricity prices for hourly settled consumers (i.e. bigger firms) respectively flex and residual settled consumers (ie. small firms and households). Estimating how price elasticities of demand of electricity has been an economic area of interest for a long time and increasingly so. 
\par 

The electricity market has changed vastly over the past few decades in the direction of more competition and a larger share of intermittent, renewable energy production capacity. The climate crisis and the related ongoing political debate suggests that this will be equally if not more important in the near future. To policy makers and voters alike decarbonization is strongly linked to greater electrification, but this will only be true if this renewable energy production is large enough to answer to demand.  Efficient and environmentally sustainable electricity provision implies that electricity production and thus electricity supply fluctuates according to weather conditions, namely wind speed and sunshine. Heterogeneity and changes over time in demand responses can help predicting potential demand flexibility in the future as this is the main limit for further increasing the reliance on wind and solar power along with the infeasibility of electricity storage. From a policy point of view this can also reveal the potential for time-of-use tariffs (and other demand responses) which are being regarded as the most cost-efficient tool for promoting a more sustainable electricity consumption cf. \citet{albadi2008summary}.
\par

\todo{"We contribute to the existing literature by ..." tilføjes} We use hourly observations since January 2016 on aggregated consumption for 52 of the local grid companies in Denmark. To account for heterogeneity across the grid companies we estimate the price-elasticity both grid-by-grid and on aggregate while controlling for grid-level specific effects. The problems of endogeneity resulting from the simultaneity of demand and supply mechanics is handled by instrumenting the hourly price by both lagged prices and the prognosis for wind power production. *** >> OUR RESULTS <<***
\par

The paper proceeds by giving a brief account of related studies in section \ref{sec:background}. Section \ref{sec:theory} covers the price formation in the electricity market by going into detail with the market itself in \ref{subsec:t_market}, the production side in \ref{subsec:t_production} and perspectives of the demand side in \ref{subsec:t_demand}. The data used for our empirical analysis is described in section \ref{sec:data} while we go into details with the econometric estimation method in section \ref{sec:empirical}. Results from the analysis are presented and discussed in \ref{sec:results}. Section \ref{sec:conclusion} concludes. 

