\label{sec:intro}
% Motivation of the study: why you focus on this particular issue
% Hypothesis and objective(s)
% Description of the background (theoretical and empirical) that lead you to propose the hypothesis
% Approach and summary of results: what is your strategy to check the hypothesis and the main result
% Structure of the paper
The focus of this paper is how the hour-by-hour electricity consumption responds to hourly electricity prices for firms and households respectively. We use hourly observations since January 2016 on aggregated consumption for 52 of the local grid companies in Denmark.
\medskip\\

To account for heterogeneity across the grid companies we estimate the price-elasticity both grid-by-grid and on aggregate while controlling for grid-level specific effects. The problems of endogeneity resulting from the simultaneity of demand and supply mechanics is handled by instrumenting the hourly price by both lagged prices and the prognosis for wind power production.
\medskip\\

*SECTION ON OUR RESULTS* 
\medskip\\

Efficient and environmentally sustainable electricity provision implies that electricity production and thus electricity supply fluctuates according to weather conditions, namely wind speed and sunshine. This \textcolor{orange}{along with changes in the distribution of electricity}\todo{is this precise?} 
implies that consumers are increasingly exposed to more volatile electricity prices. Heterogeneity and changes over time in demand responses can help predicting potential demand flexibility in the future as this is the main limit for further increasing the reliance on wind and solar power along with the infeasibility of electricity storage. From a policy point of view this can also reveal the potential for time-of-use tariffs which is often regarded as the best policy tool for incentivizing behavioural changes towards a more sustainable electricity consumption.