%%% File encoding is ISO-8859-1 (also known as Latin-1)
%%% You can use special characters just like ä,ü and ñ

% Input encoding is 'latin1' (Latin 1 - also known as ISO-8859-1)
% CTAN: http://www.ctan.org/pkg/inputenc
%
% A newer package is available - you may look into:
% \usepackage[x-iso-8859-1]{inputenc}
% CTAN: http://www.ctan.org/pkg/inputenx
\usepackage[utf8]{inputenc}
\usepackage{import}
\usepackage{dsfont}
% Font Encoding is 'T1' -- important for special characters such as Umlaute ü or ä and special characters like ñ (enje)
% CTAN: http://www.ctan.org/pkg/fontenc
\usepackage[T1]{fontenc}
%\usepackage[osf, sc]{mathpazo} % Use the Palatino font

% Language support for 'english' (alternatives: british,UKenglish,USenglish,american)
% CTAN: http://www.ctan.org/pkg/babel
\usepackage[english]{babel} %
\usepackage{csquotes}

% Extended graphics support
% There is also a package named 'graphics' - watch out!
% CTAN: http://www.ctan.org/pkg/graphicx
\usepackage{graphicx}

%customized line spacing
\usepackage{setspace}
\onehalfspacing
\usepackage{dirtytalk}
%Inclusion of pdf's
\usepackage{pdfpages}

%Create random text
\usepackage{lipsum} % \lipsum[1] or \lipsum[1-3]

% biblatex is used for creating bibliography
\usepackage[style=authoryear, backend=bibtex8, natbib=true, maxcitenames=2]{biblatex}

% More comprehensive math typing
\usepackage{amssymb}

% To allow new math operators
\usepackage{amsmath}
\usepackage{mathtools}
\usepackage{bm} % bold symbol in math mode

% Nice matrix features
\usepackage{physics}
\usepackage[ruled,vlined]{algorithm2e}
% Abbreviations
\usepackage[hyperref=true]{acro}

% Structure
\usepackage{float}
\usepackage{placeins}
\usepackage{multirow}
\usepackage{multicol}
\usepackage{pdfpages}
\usepackage{kbordermatrix}% http://www.hss.caltech.edu/~kcb/TeX/kbordermatrix.sty

% Tables, figures
\usepackage{booktabs} %Create publication quality tables
\usepackage{tabularx}
\usepackage{threeparttable}
\usepackage{siunitx}
\usepackage[figuresright]{rotating}
\usepackage{subcaption}
\usepackage{caption}
\newcommand{\sourcecenter}[1]{\vspace{-6pt} \caption*{\textit{Source: } { #1} \hfill} } % centered
\newcommand{\sourceleft}[1]{\vspace{-18pt} \caption*{\textit{Source: } { #1} \hspace*{\fill}} }  % left-aligned

% Text
\usepackage{xcolor}
\usepackage{enumerate} % numbered lists
\usepackage[super]{nth} % Write 1st, 2nd as \nth{1}, \nth{2} etc.
\usepackage{listings} % see https://tex.stackexchange.com/questions/105662/default-value-for-basicstyle-in-lstlisting

% Front page
%Add affiliations to authors name, allowing to mix affiliations
\usepackage{authblk}  % \author[a]{F. Hvam} \author[a]{S. Tarly} \author[b]{J. Snow}
                      % \affil[a]{The Citadel} \affil[b]{The Wall}
\newenvironment{acknowledgements}{ %acknowledgements for title page
  \renewcommand\abstractname{Acknowledgements}\begin{abstract}} {\end{abstract}}

% Notes
\usepackage{epigraph}
\usepackage{marginnote}
% Select what to do with command \comment:
  % \newcommand{\comment}[1]{}  %comments not showed
   \newcommand{\comment}[1]{\par {\bfseries \color{blue} #1 \par}} %comments showed
% Select what to do with todonotes: i.e. \todo{}, \todo[inline]{}
  %\usepackage[disable]{todonotes} % notes not showed
   \usepackage[draft]{todonotes}   % notes showed

% % 'Ditto' signs for tables
% \usepackage{tikz}
%   \newcommand{\ditto}{
%       \tikz{
%           \draw [line width=0.12ex] (-0.2ex,0) -- +(0,0.8ex)
%               (0.2ex,0) -- +(0,0.8ex);
%           \draw [line width=0.08ex] (-0.6ex,0.4ex) -- +(-1.0em,0)
%               (0.6ex,0.4ex) -- +(1.0em,0);
%       }
%   }
